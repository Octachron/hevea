\documentstyle{article}

\begin{document}

\section{Test de base\ldots}

Test {\tt tt \it \Large tt it Large {\bf tt it Large bf } tt it Large
{\rm Large rm } et encore }

{\Large Large rm}

{\Large\textsc{\texttt{\textrm{Roman, small caps}}}}

{\Large\textsf{\textit{Slanted familly, italic shape} Slanted familly}}

Test des exposants abusifs $1^{2^{3^{4^5}}}$



\section{Test du display}
$$
 1 + {\cal A} + 1 = 3
$$

\section{Les tableaux}

\subsection{Environnement d�finissant un tableau}

\newenvironment{coucou}{\begin{tabular}{cc}}{\end{tabular}}

\begin{coucou}
1 & 2
\end{coucou}

\subsection{Tableaux d'une ligne}
$$
\begin{array}{ll}
1 & 2
\end{array}
$$

$$
\left(\begin{array}{c} 1 \\ \begin{array}{ll}2 & 3\end{array}\end{array}\right)
$$

\subsection{Multicolonnes}

\begin{tabular}{|rl}
  zobi & zobi \\
\multicolumn{2}{c}{coucou-coucou} \\
\multicolumn{1}{c}{zobi-1} & \multicolumn{1}{c}{zobi-2} 
\end{tabular}


\begin{center}
\begin{tabular}{|r|rl|l|} \hline
\multicolumn{1}{|c}{$n$} & \multicolumn{1}{|c}{$N$} &  & 
                                             \multicolumn{1}{|c|}{Exemple}\\
\hline
16 & 65 536                      &  $= \; 6 \times 10^4$  & Macintosh SE/30 \\
\hline
32 & 4 294 967 296               &  $= \; 4 \times 10^9$  & Sun, Hp \\
\hline
64 & 18 446 744 073 709 551 616  &  $ = \; 2 \times 10^{19}$ & Alpha \\
\hline
\end{tabular}
\end{center}

\subsection{Les d�limiteurs}
$$
\left\{\begin{array}{cc}
\left(\begin{array}{c}1\end{array}\right) &
\left(2\right)
\end{array}\right\}
$$

$$\begin{array}{lll}\hline \\ \hline \\
\left(\begin{array}{rrr}
4 & 9 & 2 \\
3 & 5 & 7 \\
8 & 1 & 6
\end{array} \right) &
\left(\begin{array}{rrrrr}
11 & 18 & 25 &  2 & 9 \\
10 & 12 & 19 & 21 & 3 \\
 4 &  6 & 13 & 20 & 22 \\
23 &  5 &  7 & 14 & 16 \\
17 & 24 &  1 &  8 & 15
\end{array} \right) &
\left(\begin{array}{rrrrrrc}
22 & 31 & 40 & 49 &  2 & 11 & 20 \\
21 & 23 & 32 & 41 & 43 &  3 & 12 \\
13 & 15 & 24 & 33 & 42 & 44 &  4 \\
 5 & 14 & 16 & 25 & 34 & 36 & 45 \\
46 &  6 &  8 & 17 & 26 & 35 & 37 \\
38 & 47 &  7 &  9 & 18 & 27 & 29 \\
30 & 39 & 48 &  1 & 10 & 19 &
\left\{\begin{array}{ll} 0 & 0 \\ 0 & 0 \end{array}\right\}
\end{array} \right) 
\end{array} $$

\subsection{Un exemple tr�s tordu}

\begin{center}
\begin{tabular}{lllllllllllll}
% +
\begin{tabular}{|@{\hspace{.5ex}}c@{\hspace{.5ex}}|}
    \\
    \\
    \\
    \\
    \\
$+$ \\ \hline
\end{tabular} &
% + *
\begin{tabular}{|@{\hspace{.5ex}}c@{\hspace{.5ex}}|}
    \\
    \\
    \\
    \\
$*$ \\
$+$ \\ \hline
\end{tabular} &
% + * +
\begin{tabular}{|@{\hspace{.5ex}}c@{\hspace{.5ex}}|}
    \\
    \\
    \\
$+$ \\
$*$ \\
$+$ \\ \hline
\end{tabular} &
% + * + 35
\begin{tabular}{|@{\hspace{.5ex}}c@{\hspace{.5ex}}|}
    \\
    \\
$35$ \\
$+$   \\
$*$ \\
$+$ \\ \hline
\end{tabular} &
% + * 71
\begin{tabular}{|@{\hspace{.5ex}}c@{\hspace{.5ex}}|}
    \\
   \\
    \\
71   \\
$*$ \\
$+$ \\ \hline
\end{tabular} &
% + * 71 +
\begin{tabular}{|@{\hspace{.5ex}}c@{\hspace{.5ex}}|}
    \\
    \\
$+$    \\
71    \\
$*$  \\
$+$ \\ \hline
\end{tabular} &
% + * 71 + 5
\begin{tabular}{|@{\hspace{.5ex}}c@{\hspace{.5ex}}|}
     \\
 5   \\
$+$    \\
71    \\
$*$  \\
$+$ \\ \hline   
\end{tabular} &
% + 781
\begin{tabular}{|@{\hspace{.5ex}}c@{\hspace{.5ex}}|}
    \\
    \\
    \\
    \\
781  \\
$+$ \\ \hline
\end{tabular} &
% + 781 *
\begin{tabular}{|@{\hspace{.5ex}}c@{\hspace{.5ex}}|}
    \\
    \\
    \\
$*$    \\
781  \\
$+$ \\ \hline
\end{tabular} & 
% + 781 * +
\begin{tabular}{|@{\hspace{.5ex}}c@{\hspace{.5ex}}|}
    \\
  \\
$+$ \\
$*$ \\
781  \\
$+$ \\ \hline
\end{tabular} &
% + 781 * + 7
\begin{tabular}{|@{\hspace{.5ex}}c@{\hspace{.5ex}}|}
    \\
 7   \\
 $+$ \\
$*$ \\
781  \\
$+$ \\ \hline
\end{tabular} &
% + 781 * 15
\begin{tabular}{|@{\hspace{.5ex}}c@{\hspace{.5ex}}|}
    \\
$*$ \\
15 \\
$*$ \\
781  \\
$+$ \\ \hline
\end{tabular} &
% + 781 * 15 * 9* 9
\begin{tabular}{|@{\hspace{.5ex}}c@{\hspace{.5ex}}|}
9   \\
$*$ \\
15 \\
$*$ \\
781  \\
$+$ \\ \hline
\end{tabular} 
% 1996
\end{tabular}
\end{center}

\subsection{Les exposants terribles}\label{plushaut}
$$
 2^{\left.\begin{array}{@{\hspace{0ex}}c@{\hspace{0ex}}}
{\mbox{\scriptsize 2}^{\mbox{$\cdot^{\mbox{$\cdot^{\mbox{$\cdot^2$}}$}}$}}
}\end{array}\right\} n}
$$

\subsection{Newtheorm + cases}

\newtheorem{th}{Th\'eor\`eme}
\newtheorem{ath}[th]{Autre th�or�me}
\newcommand{\Preuve}{\noindent{\bf Preuve} \hspace{1ex}}
\begin{th}\label{un}
Soit $M^p$ la puissance p-i\`eme de la matrice $M$, le co{e}fficient
$M^{p}_{i,j}$ est \'egal au nombre de chemins de longueur $p$ de $G$
dont l'origine est le sommet $x_i$ et dont l'extr\'emit\'e est le
sommet $x_j$.
\end{th}

\begin{ath}[Pour voir]\label{deux}
�a devrait avoir le num�ro 2.
\end{ath}
\Preuve On effectue une r\'ecurrence sur $p$. Pour $p=1$ le
r\'esultat est imm\'ediat car un chemin de longueur $1$ est un arc du
graphe. Le calcul de $M^p$, pour $p > 1$ donne: \[M^{p}_{i,j} =
\sum_{k=1}^{n}M^{p-1}_{i,k} M_{k,j}\] Or tout chemin de longueur $p$
entre $x_i$ et $x_j$ se d\'ecompose en un chemin de longueur $p-1$
entre $x_i$ et un certain $x_k$ suivi d'un arc reliant $x_k$ et $x_j$.
Le r\'esultat d\'ecoule alors de l' hypoth\`ese de r\'ecurrence
suivant laquelle $M^{p-1}_{i,k}$ est le nombre de chemins de longueur
$p-1$ joignant $x_i$ \`a $x_k$.

$$L(i,j)=\cases{1 + L(i-1,j-1)            &si $a_i=b_j$ \\
                \max (L(i,j-1),L(i-1,j))  &sinon.}      \eqno{(*)}$$

\section{Quelques r�f�rences}\label{ici}
Alors ici je suis en~\ref{ici}. Ensuite je serai l�-bas
en~\ref{labas}.
Il y a d�j� un petit moment j'�tais plus haut avec plein de potes exposants~\ref{plushaut}.
J'ai aussi �crit des th�or�mes~\ref{un} et~\ref{deux}.

J'essaie aussi une figure~\ref{figure}
\begin{figure}[t]
\begin{center}Pan dans la figure\end{center}
\caption{\label{figure}Je suis une figure}
\end{figure}

\section{Quelques notes de bas de page}\label{labas}
Coucou\footnote{coucou}
\textsl{et l� ??\footnote{Une autre}}.

\end{document}