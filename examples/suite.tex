\documentclass{article}
\usepackage{latexsym}
\usepackage{isolatin1}
\usepackage{../hevea}
\usepackage{index}
\usepackage{color}

\title{Test de \hevea\footnote{Une �toile}}
\makeindex

\begin{document}
\maketitle

\section{Test des polices}

Test {\tt tt \it \Large tt it Large {\bf tt it Large bf } tt it Large
{\rm Large rm } et encore }

\begin{center}\Large
Roman

\textsc{\texttt{\textrm{Roman, small caps}}}

\textsf{\textit{Sans Sherif family, italic shape} Sans Sherif family}

\texttt{\textsl{TT family, slanted shape}}

\end{center}

\begin{htmlonly}
\subsection{Couleurs � la \hevea{}}
{\Huge
{\purple purple}
{\silver silver}
{\gray gray}
{\white white}
{\maroon maroon}
{\red red}
{\fuchsia fuchsia}
{\green green}
{\lime lime}
{\olive olive}
{\yellow yellow}
{\navy navy}
{\blue blue}
{\teal teal}
{\aqua aqua}
{\htmlcolor{202020}et \c{c}a, c'est presque noir~?}
{\htmlcolor{e0e0e0}et \c{c}a, c'est presque blanc~?}
}
\end{htmlonly}

\subsection{Couleurs � la \textit{color}}
\textcolor{black}{noir},
\textcolor{white}{blanc},
\textcolor{red}{rouge},
\textcolor{green}{vert},
\textcolor{blue}{bleu},
\textcolor{cyan}{cyan},
\textcolor{yellow}{jaune},
\textcolor{magenta}{magenta}.

\definecolor{medium}{gray}{0.5}
\textcolor{medium}{gris moyen}.

\definecolor{otheryellow}{cmyk}{0, 0, 1, 0.5}
\textcolor{otheryellow}{Un autre jaune}.

\section{Paragraphs}

\subsection{No par skip}
coucou \begin{em}coucou\end{em} ha ??
\begin{flushleft}
\quad\verb+\begin{list}{+{\it default\_label}\verb+}{+{\it decls}\verb+}+
\end{flushleft}
coucou
\begin{verbatim}
coucou
\end{verbatim}
Coucou
\begin{quote}
Quotation
\end{quote}
Coucou
$$
\left(\begin{array}{ll}
1 & 2
\end{array}\right)
\qquad
\begin{array}{ll}
1 & 2
\end{array}
$$
Yet another
\subsection{Par skip}
coucou \begin{em}coucou\end{em} ha ??

\begin{flushleft}
\quad\verb+\begin{list}{+{\it default\_label}\verb+}{+{\it decls}\verb+}+
\end{flushleft}

coucou

\begin{verbatim}
coucou
\end{verbatim}

Coucou

\begin{quote}
Quotation.
\end{quote}

Coucou

$$
\left(\begin{array}{ll}
1 & 2
\end{array}\right)
\qquad
\begin{array}{ll}
1 & 2
\end{array}
$$

Yet another

\subsection{Paragraphes dans les listes imbriqu�es (du �
Ph. A. Vitton)}

This is a test of nested lists.
\begin{itemize}
\item This the first item in the top-level

\item This is the second item in the top-level

\begin{itemize}
\item This is the first item in the second level

\item This is the second item in the second level
\end{itemize}

\item This is the last item in the top level
\end{itemize}
And here is the end of the test.

\section{Arguments}
Le i chap� sans espaces~: ``\^\i \^{\i }\^\i\^{\i}''

Le i chap� avec espaces~: ``\^\i{} {\^{\i }} \^\i\ \^{\i}''

Les caract\`eres accentu\'es sans peine~:
``bien \'elev\'e bien \'{e}lev\'{e} tr\`es bien.''

Chaque tableau est un paragraphe
\newcommand{\test}[2]{\begin{tabular}{rcl}arg$_1$ & = & ``#1'' \\ arg$_2$ &
= & ``#2''\end{tabular}}
\newcommand{\aaa}{$\alpha$}

\test{a}{b}

\test\aaa\aaa

\test\aaa{\aaa}

\test\i i
\section{Math mode}

\subsection{Italiques en mode math}

Alors j'y vais~: $1 + a_{i+1} + \int^N_01/x dx + \frac{\partial x}{2}$

Et puis une petite mbox en exposant~: {\LARGE $\alpha^{\mbox{gros $\beta$}}$}.

Et puis faut bien essayer les autres trucs:

\begin{tabular}{ll}
{\bf bold} & $\mathbf{1 + a_{i+1} + \int^N_01/x dx + \frac{\partial
x}{2}}$\\
{\sf sans sherif} &
$\mathsf{1 + a_{i+1} + \int^N_01/x dx + \frac{\partial x}{2}}$\\
{\it italique} &
$\mathit{1 + a_{i+1} + \int^N_01/x dx + \frac{\partial x}{2}}$
\end{tabular}

\noindent Bon, on a connu pire~!


\subsection{Test des exposants abusifs} $1^{2^{3^{4^5}}}$



\section{Test du display}
$$
 1 + {\cal A} + 1 = 3
$$

Comme au dessus mais en mode display:

$$1 + a_{i+1} + \int^N_01/x dx + \frac{\partial x}{2}$$

Mauvais espaceemnt en mode display~:
$$
{C_N \over C_{N-1}} + \frac{C_{N+1}}{C_N} + \begin{array}{c}C_{N-2} \\
\hline C_{N-1}\end{array} + 4 + \cdots + 5
$$

\section{Les fractions}
\[
{x\over y} + f+ \frac{a}{b} + 1
\]

$$
A \over B
$$
\subsection{Exemples d'utilisateurs}

Ce sont des m�langes de fractions et de d�limiteurs.
\subsection{L�vy}
{\bfseries
$\begin{array}{@{\hspace{0ex}}l@{\hspace{0.6ex}}l}
 {C_N \over {N+1}} & = {C_{N-1} \over N} + {2 \over {N+1}} = {C_2 \over 3} +
\sum_{3 \leq k \leq N} {2 \over {k+1}} \\[1.7ex]

                  &\simeq 2 \sum_{1 \leq k \leq N} {1 \over k} \simeq
2 \int_1^N {1 \over x} dx =  2 \ln N 
\end{array}$
}
\subsection{Brisset}
$$
\left\{{b\over\left({c\over{}d}\right)}\right\}
$$

\subsection{Alliot}
$$
k=
\left(
\tan {\left[
\left(
\frac{n}{N+1}%
\right)
\frac{\pi}2
\right]}%
\right)^{p}
$$
\subsection{Maranget}
$$
\sum_{t=2}^T u_{it} + \frac{1}{\frac{1}{1 + \frac{\pi}{2}}}
$$

\section{Les tableaux}

\subsection{Environnement d�finissant un tableau}

\newenvironment{tabdeux}{\begin{tabular}{cc}}{\end{tabular}}

\begin{tabdeux}
1 & 2
\end{tabdeux}

\subsection{Les modes dans les tableaux}

\begin{tabular}{c@{foo}c}
\tt tt & \it it\\
\sf sf & \sc sc\\
\end{tabular}

\subsection{Tableaux d'une ligne}
$$
\begin{array}{ll}
1 & 2
\end{array}
$$

$$
\left(\begin{array}{c} 1 \\
\begin{array}{ll}
2 & 3
\end{array}
\end{array}\right)
$$

\begin{tabular}{|*{10}{c}}
1 & 2 & 3 & 4 & 5 & 6 & 7 & 8 & 9 & 10 \\
3 & 4 & 5 & 6 & 7 & 8 & 10 & 9 & 1 & 2
\end{tabular}

\subsection{Multicolonnes}

\begin{tabular}{|rl}
  zobi & zobi \\
\multicolumn{2}{c}{coucou-coucou} \\
\multicolumn{1}{c}{zobi-1} & \multicolumn{1}{c}{zobi-2} 
\end{tabular}

\begin{center}
\newcommand{\sep}{\quad}
\begin{tabular}{l|@{\sep}r@{\sep}r@{\sep}r@{\sep}r@{\sep}r}
& \texttt{fib} &
\texttt{afib} &
\texttt{pat} &
\texttt{qsort} &
\texttt{count}\\ \hline
\texttt{join} &  32.0 & 14.5 & 37.2 & 9.9 & 16.4 \\
\texttt{jocaml} & 5.7 &  3.5 & 5.4 & 1.4 & 4.2 \\
\texttt{Bologna} & 11.9 & 6.2 &  9.4 & 16.8 & 5.3 \\
\end{tabular}
\end{center}

\begin{center}
\begin{tabular}{|r|rl|l|} \hline
\multicolumn{1}{|c}{$n$} & \multicolumn{1}{|c}{$N$} &  & 
                                             \multicolumn{1}{|c|}{Exemple}\\
\hline
16 & 65 536                      &  $= \; 6 \times 10^4$  & Macintosh SE/30 \\
\hline
32 & 4 294 967 296               &  $= \; 4 \times 10^9$  & Sun, Hp \\
\hline
64 & 18 446 744 073 709 551 616  &  $ = \; 2 \times 10^{19}$ & Alpha \\
\hline
\end{tabular}
\end{center}


\begin{tabular}{l@{\qquad}p{.3\linewidth}@{\qquad}p{.3\linewidth}}
Macro & \multicolumn{1}{c}{\hevea} &  \multicolumn{1}{c}{\LaTeX}\\
\hline

\verb+\url{+\textit{url}\verb+}{+\textit{text}\verb+}+ &
make \textit{text} an hyperlink to \textit{url} &
echo \textit{text}\\ \hline

\verb+\footurl{+\textit{url}\verb+}{+\textit{text}\verb+}+ &
make \textit{text} an hyperlink to \textit{url} &
make \textit{url} a footnote to \textit{text},
\textit{url} is shown in typewriter font\\ \hline

\verb+\oneurl{+\textit{url}\verb+}+ &
make \textit{url} an hyperlink to \textit{url}.
&
typeset \textit{url} in typewriter font\\ \hline
\verb\\

\verb+\mailto{+address\verb+}+ &
make \textit{address} a ``mailto'' link to \textit{address} &
typeset \textit{address} in typewriter font\\ \hline

\verb+\home{+\textit{text}\verb+}+ &
\multicolumn{2}{p{.6\linewidth}}{produce a home-dir url both for output and links, output aspect is: ``\home{\textit{text}}''}
\end{tabular}

\subsection{Les d�limiteurs}
$$
\left\{\begin{array}{cc}
\left(\begin{array}{c}1\end{array}\right) &
\left(2\right)
\end{array}\right\}
$$

$$
\left|\left\{
\left( 1 \right) =
\left( 1 \right) =
\left( 1 \right) =
\left[\begin{array}{c}1\\2\\3\end{array}\right]
\right\}\right\}
$$

$$\begin{array}{lll}\hline \\ \hline \\
\left(\begin{array}{rrr}
4 & 9 & 2 \\
3 & 5 & 7 \\
8 & 1 & 6
\end{array} \right) &
\left(\begin{array}{rrrrr}
11 & 18 & 25 &  2 & 9 \\
10 & 12 & 19 & 21 & 3 \\
 4 &  6 & 13 & 20 & 22 \\
23 &  5 &  7 & 14 & 16 \\
17 & 24 &  1 &  8 & 15
\end{array} \right) &
\left(\begin{array}{rrrrrrc}
22 & 31 & 40 & 49 &  2 & 11 & 20 \\
21 & 23 & 32 & 41 & 43 &  3 & 12 \\
13 & 15 & 24 & 33 & 42 & 44 &  4 \\
 5 & 14 & 16 & 25 & 34 & 36 & 45 \\
46 &  6 &  8 & 17 & 26 & 35 & 37 \\
38 & 47 &  7 &  9 & 18 & 27 & 29 \\
30 & 39 & 48 &  1 & 10 & 19 &
\left\{\begin{array}{ll} 0 & 0 \\ 0 & 0 \end{array}\right\}
\end{array}\right) 
\end{array} $$

\subsection{Un exemple tr�s tordu}

\begin{center}
\begin{tabular}{lllllllllllll}
% +
\begin{tabular}{|@{\enspace}c@{\enspace}|}
    \\
    \\
    \\
    \\
    \\
$+$ \\ \hline
\end{tabular} &
% + *
\begin{tabular}{|@{\enspace}c@{\enspace}|}
    \\
    \\
    \\
    \\
$*$ \\
$+$ \\ \hline
\end{tabular} &
% + * +
\begin{tabular}{|@{\enspace}c@{\enspace}|}
    \\
    \\
    \\
$+$ \\
$*$ \\
$+$ \\ \hline
\end{tabular} &
% + * + 35
\begin{tabular}{|@{\enspace}c@{\enspace}|}
    \\
    \\
$35$ \\
$+$   \\
$*$ \\
$+$ \\ \hline
\end{tabular} &
% + * 71
\begin{tabular}{|@{\enspace}c@{\enspace}|}
    \\
   \\
    \\
71   \\
$*$ \\
$+$ \\ \hline
\end{tabular} &
% + * 71 +
\begin{tabular}{|@{\enspace}c@{\enspace}|}
    \\
    \\
$+$    \\
71    \\
$*$  \\
$+$ \\ \hline
\end{tabular} &
% + * 71 + 5
\begin{tabular}{|@{\enspace}c@{\enspace}|}
     \\
 5   \\
$+$    \\
71    \\
$*$  \\
$+$ \\ \hline   
\end{tabular} &
% + 781
\begin{tabular}{|@{\enspace}c@{\enspace}|}
    \\
    \\
    \\
    \\
781  \\
$+$ \\ \hline
\end{tabular} &
% + 781 *
\begin{tabular}{|@{\enspace}c@{\enspace}|}
    \\
    \\
    \\
$*$    \\
781  \\
$+$ \\ \hline
\end{tabular} & 
% + 781 * +
\begin{tabular}{|@{\enspace}c@{\enspace}|}
    \\
  \\
$+$ \\
$*$ \\
781  \\
$+$ \\ \hline
\end{tabular} &
% + 781 * + 7
\begin{tabular}{|@{\enspace}c@{\enspace}|}
    \\
 7   \\
 $+$ \\
$*$ \\
781  \\
$+$ \\ \hline
\end{tabular} &
% + 781 * 15
\begin{tabular}{|@{\enspace}c@{\enspace}|}
    \\
$*$ \\
15 \\
$*$ \\
781  \\
$+$ \\ \hline
\end{tabular} &
% + 781 * 15 * 9* 9
\begin{tabular}{|@{\enspace}c@{\enspace}|}
9   \\
$*$ \\
15 \\
$*$ \\
781  \\
$+$ \\ \hline
\end{tabular} 
% 1996
\end{tabular}
\end{center}



\subsubsection{Quelques d�limiteurs imbriqu�s}

\begin{table}[h]
$$
\left[\left(\begin{array}{cc} 1 & 2 \\ \alpha & a_i^k
\end{array}\right)
+ \sum_{j=0}^{\kappa(j) < \Delta(k)} \left|\begin{array}{cc} {\cal A}  & \aleph \\ \gamma & \partial
x\over y \end{array}\right|\right]
$$
\caption{\label{t1}Les d�limiteurs dans une table (voir section~\ref{allrefs})}
\end{table}

\subsection{Les exposants terribles}\label{plushaut}
$$
 2^{\left.\begin{array}{@{}c@{}}
{\mbox{\scriptsize 2}^{\mbox{$\cdot^{\mbox{$\cdot^{\mbox{$\cdot^2$}}$}}$}}
}\end{array}\right\} n}
$$

\subsection{Newtheorm + cases}

\newtheorem{theoreme}{Th\'eor\`eme}
\newtheorem{ath}[theoreme]{Autre th�or�me}
\newcommand{\Preuve}{\noindent{\bf Preuve} ~}

\begin{theoreme}\label{un}
Soit $M^p$ la puissance p-i\`eme de la matrice $M$, le co{e}fficient
$M^{p}_{i,j}$ est \'egal au nombre de chemins de longueur $p$ de $G$
dont l'origine est le sommet $x_i$ et dont l'extr\'emit\'e est le
sommet $x_j$.
\end{theoreme}
\begin{ath}[Pour voir]\label{deux}
�a devrait avoir le num�ro 2.
\end{ath}
\Preuve On effectue une r\'ecurrence sur $p$. Pour $p=1$ le
r\'esultat est imm\'ediat car un chemin de longueur $1$ est un arc du
graphe. Le calcul de $M^p$, pour $p > 1$ donne: \[M^{p}_{i,j} =
\sum_{k=1}^{n}M^{p-1}_{i,k} M_{k,j}\] Or tout chemin de longueur $p$
entre $x_i$ et $x_j$ se d\'ecompose en un chemin de longueur $p-1$
entre $x_i$ et un certain $x_k$ suivi d'un arc reliant $x_k$ et $x_j$.
Le r\'esultat d\'ecoule alors de l' hypoth\`ese de r\'ecurrence
suivant laquelle $M^{p-1}_{i,k}$ est le nombre de chemins de longueur
$p-1$ joignant $x_i$ \`a $x_k$.

$$L(i,j)=\cases{1 + L(i-1,j-1)            &si $a_i=b_j$ \cr
                \max (L(i,j-1),L(i-1,j))  &sinon.}      \eqno{(*)}
$$

\section{Quelques r�f�rences}\label{ici}\label{allrefs}
Alors ici je suis en~\ref{ici}. Ensuite je serai l�-bas
en~\ref{labas}.
Il y a d�j� un petit moment j'�tais plus haut avec plein de potes exposants~\ref{plushaut}.
J'ai aussi �crit des th�or�mes~\ref{un} et~\ref{deux}.

J'ai d�j� fait une table~: la table~\ref{t1}. J'en ai une autre � la
fin, la table~\ref{t2}.

J'essaie aussi une figure~\ref{figure}
\begin{figure}[t]
\begin{center}Pan dans la figure\end{center}
\caption{\label{figure}Je suis une figure}
\end{figure}

\section{Quelques notes de bas de page}\label{labas}
Coucou\footnote{coucou}
\textsl{et l� ??\footnote{Une autre}}.
Une note recompos�e\footnotemark\footnotetext{Num�ro trois}.

\begin{table*}
\begin{flushleft}
Ceci est la table de la fin.\\
Bref c'est fini.\\
\end{flushleft}
\caption{\label{t2}}
\end{table*}

\section{Encore un effort}

\def\trans#1{\stackrel{#1}{\rightarrow}}
$$
A_0 = \epsilon(\{e_0\}) \trans{\alpha_0} A_1 \trans{\alpha_1} \cdots \trans{\alpha_{k-1}} A_k
$$

$$
e_i \in \epsilon(A) ~ \mbox{et} ~ (e_i,e_{i+1},\epsilon) \in T  \Rightarrow
e_{i+1} \in \epsilon(A)
$$

$$
A^{3+k-1}_{j+8-i}
$$

$$\tt
\alpha_j
$$

\begin{itemize}\label{coucou}
\item Chouette r\'ef\'erence~\ref{coucou}, non ?
\end{itemize}

\section{Quelques listes}

\begin{trivlist}
\item Coucou
\item Zobi
\end{trivlist}

\newcounter{coucou}
\renewcommand{\thecoucou}{\Alph{coucou}}
\begin{list}{\thecoucou}{\renewcommand{\makelabel}[1]{\underline{#1}}\usecounter{coucou}}
\item Coucou
\item Zobi
\end{list}

\newenvironment{foo}{\begin{trivlist}\it\item[coucou]}{\end{trivlist}}
\begin{foo}
Foo! foooooooooooooooooooo.
\end{foo}


\section{Beurk}

{\centering coucou}

{\raggedleft coucou}

{\raggedright coucou}

\begin{center}
coucou
\end{center}

\begin{flushleft}
coucou
\end{flushleft}

\begin{flushright}
coucou
\end{flushright}


\section{Test de l'index}
\newcommand{\gros}[1]{{\Huge#1}}
\index{a|gros}
\index{a@$\alpha$}
\index{b}
\index{b!a}
\index{b!a!a}
\index{b!a!a}
\index{b!a!a!a|gros}
\index{b!a!a!a}
\index{b!a!a!a!a!a!a|see{ailleurs}}
\index{b!a!a!a!a!a!a|textit}
\index{b!a!a!a@$\alpha_{\alpha}$}
\index{b!a!a!b}
\index{b!a!b}
\index{b@$\beta$|see{b}}
\index{foo}

Dans cet index les num�ros doivent �tre croissants.
\printindex
\section{Macro substitution}

\noindent X\enspace X\\
X\quad X\\
X\qquad X

%\newcommand{\app}[2]{#1{#2}}
%\app{\texttt}{coucou}

\newcommand{\gensymbol}{{\rm\sl symbol}}
\newcommand{\nelist}[2]{#1 {\tt #2} \ldots {\tt #2} #1}


Un espace~: ``\nelist{\gensymbol}{sep}''

Tout coll�~: \gensymbol {\tt sep}

Les arguments substitu�s sont lex�s sans tenir compte de la suite
\newcommand{\coucou}[2]{{#1#2}}
\newcommand{\coucoubis}[2]{{#1 #2}}

Ca marche~: \coucou{\it}{zobi}

Il y a un espace avant ``zobi''~: [\coucoubis{\it}{zobi}]
\subsection{un}

Tout pareil:
\begin{quote}
$1+2^n$
\(1+2^n\)
\begin{math}1+2^n\end{math}
\ensuremath{1+2^n}
$\ensuremath{1+2^n}$
\end{quote}

Tout pareil en display:
\begin{quote}
$$ 1 + \sum_{i=0}^{2^n} i^i $$
\begin{displaymath}
1 + \sum_{i=0}^{2^n} i^i
\end{displaymath}
\end{quote}

\begin{equation}
\pi \sim 3.14159\ldots\label{uneq}
\end{equation}

\subsection{deux}
\begin{equation}\label{deuxeq}
e \sim 2.71\ldots
\end{equation}

J'ai fait deux �quations~\ref{uneq} et~\ref{deuxeq}.
J'en fais encore deux ci-dessous~\ref{troiseq} et~\ref{cinqeq}:
\begin{eqnarray}
e & = & \sum_{n=0}^{\infty} \frac{1}{n!}\label{troiseq}\\
\nonumber\pi & \sim & 4 * \arctan(1)\label{quatreeq}\\
\log(1+\epsilon) & = & 1 + \epsilon + O(\epsilon)\label{cinqeq}\\
\nonumber\lefteqn{1 + 2 + 3 + \cdots + n}\\
  & = & \frac{n*(n+1)}{2}\\
\end{eqnarray}

Les m�mes sans les num�ros:
\begin{eqnarray*}
e & = & \sum_{n=0}^{\infty} \frac{1}{n!}\\
\pi & \sim & 4 * \arctan(1)
\end{eqnarray*}

\subsection{Display dans les tableaux}
Ici, \verb+array+ ouvre le mode ``display maths''. Comparons,
$\left[\sum_{n=0}^{\infty} \frac{1}{n!} + \frac{1}{2}\right]$ et
$
\left[\begin{array}{c}
 \sum_{n=0}^{\infty} \frac{1}{n!} + \frac{1}{2}
\end{array}\right]
$
Et puis aussi
$$
\left[\begin{array}{c}
\displaystyle \sum_{n=0}^{\infty} \frac{1}{n!} + \frac{1}{2}
\end{array}\right]
$$
\subsubsection{Ellipsis}
$$
1 + 2 + \cdots + n \quad
\left\{\begin{array}{c}
1\\ \vdots \\ 2
\end{array}\right\}
\quad
\left|\begin{array}{ccc}
1 & & \\ & \ddots & \\ & & 2
\end{array}\right|
$$

\subsubsection{Replacemebt symbol}
$$
%HEVEA\newcommand{\leadsto}{{\red\rightarrow}}
A \leadsto B
$$

\subsubsection{Putting one thing above the other}
$$
\overline{A} \stackrel{B}{\rightarrow} \underline{C}
$$
Et pis en texte: $\overline{A} \stackrel{B}{\rightarrow}
\underline{C}$


\subsection{Arrays}
\begin{tabular}{*{3}{l}}
1 & 2 & 3\\
\\
3 & 2 & 1
\end{tabular}

\begin{tabular*}{10cm}{*{3}{l}}
1 & 2 & 3\\
\\
3 & 2 & 1
\end{tabular*}

\end{document}