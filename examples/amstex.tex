\documentclass{article}
\usepackage{a4wide}
\usepackage{amsmath}
\usepackage{ifthen}
\usepackage{calc}

\title{Test de primitives amstex}
\date{}
\numberwithin{equation}{section}
\renewcommand{\theequation}{\thesection.\alph{equation}}

\begin{document}
\maketitle

\section{Matrices}
\section*{Toutes les r\'ef\'erences}
Les num\'eros sont~:

\begin{tabular}{*{5}{|c}|}\hline
\ref{bigmat} & \ref{dessus} & \ref{dessous} & \ref{t_2_formula}\\ \hline
\end{tabular}

\subsection{Simple}
\begin{gather}
\begin{matrix} 1 & 0\\ 0 & 1\end{matrix}
\quad
\begin{pmatrix} 1 & 0\\ 0 & 1\end{pmatrix}
\quad
\begin{Vmatrix} 1 & 0\\ 0 & 1\\ 1 & 2\end{Vmatrix}
\end{gather}

\subsection{Compliqu\'e}
\setcounter{MaxMatrixCols}{20}
\newcounter{x}
\newcommand{\row}[1]{%
\hdotsfor{#1} &
\setcounter{x}{#1}\addtocounter{x}{1}\thex
\setcounter{x}{-\value{x}}\addtocounter{x}{20} &
\hdotsfor{\value{x}}}
\begin{equation}
\begin{pmatrix}
1 & \hdotsfor{19}\\
\row{1}\\
\row{2}\\
\row{3}\\
\row{4}\\
\row{5}\\
\row{6}\\
\row{7}\\
\row{8}\\
\row{9}\\
\row{10}\\
\row{11}\\
\row{12}\\
\row{13}\\
\row{14}\\
\row{15}\\
\row{16}\\
\row{17}\\
\row{18}\\
\hdotsfor{19} & 20
\end{pmatrix}\label{bigmat}
\end{equation}

\section{Environement \texttt{cases}}
To summarize, we obtain the following exact representation for the
inverse of~$x^2e^x+1$:
\[\begin{cases}
Y(x) = y(\log x),\quad&\text{$y_{\phantom{0}}$ inverse of~$2\log
x+x+\log(1+e^{-x}/x^2)$},\\
y[x+\log(1+y_0^{-2}(x)e^{-y_0(x)})] = y_0(x),\quad&\text{$y_0$
inverse of~$x+2\log x$,}\\
y_0(x) = y_1(\log x),\quad&\text{$y_1$ inverse of~$\log x+\log(1+2\log x/x)$},\\
y_1(x) = \exp(y_2(x)),\quad&\text{$y_2$ inverse of~$x+\log(1+2xe^{-x})$},\\
y_2[x+\log(1+2y_3(x)e^{-y_3(x)})] = y_3(x),&\text{$y_3$ inverse
of~$x$.}
  \end{cases}
\]

\section{Test d'alignement}
Now the $\phi_i$s are very easy to compute:
\begin{align*}
\phi_1&  =  y_0 = 1/t_2,\\
\phi_2&  = \phi_1(y_0(x+g))-\phi_1 = {t_3t_2^2\over(1+2t_2)}
-{1+4t_2+2t_2^2\over2(1+2t_2)^3}t_2^4t_3^2+O(t_3^3),\\
\phi_3&  = \phi_2(y_0(x+g))-\phi_2 = -{1+4t_2+2t_2^2\over(1+2t_2)^3}t_2^4t_3^2+O(t_3^3),
\end{align*}

A similar treatment applies to~(\ref{bigmat}), and leads to
\begin{alignat*}{5}
x+\log(1+2xe^{-x})&  =  1/t_1(y_3(x+g))&&  =  x+2xe^{-x}-2x^2e^{-2x}+O(x^3e^{-3x}),\\
\exp[-x+\log(1+2xe^{-x})]&  =  t_2(y_3(x+g))&&  =  e^{-x}-2xe^{-2x}+4x^2e^{-3x}+O(x^3e^{-4x}).
\end{alignat*}

\subsection{\texttt{gather}, \texttt{multline}}
or one of the following (successive) refinements:
\begin{gather}\label{dessus}
\exp(e^U)\left[1-\frac{2e^{-U^{1/2}}}{U^{1/2}+4}
+\frac{2e^{-2U^{1/2}}}{(U^{1/2}+4)^2}
-\frac{4}{3}\frac{e^{-3U^{1/2}}}{(U^{1/2}+4)^3}+O(e^{-4U^{1/2}})\right],\\
\exp(e^U)\exp\left[-\frac{2e^{-U^{1/2}}}{U^{1/2}+4}\right]\left[1+
\frac{8-2U^{-1/2}-U^{-1}+U^{-3/2}}{(4+U^{-1/2})^3}
e^{-U-2U^{1/2}}+O(e^{-2U})\right],\label{dessous}
\end{gather}

Cette \'equation a le num\'ero~\ref{t_2_formula}, celles d'au dessus les
num\'ero~\ref{dessus} et~\ref{dessous}.
\begin{multline}	\label{t_2_formula}
1/t_2(y_0(x+g)) = y_0(x+\log(1+y_0^{-2}e^{-y_0}))\\
 = y_0+{e^{-y_0}\over y_0^2(1+2/y_0)}
-{1+4/y_0+2/y_0^2\over 2y_0^4(1+2/y_0)^3}e^{-2y_0}+O(e^{-3y_0}).
\end{multline}
\end{document}
