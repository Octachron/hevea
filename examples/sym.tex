\documentclass{article}
\usepackage{latexsym}
\usepackage{textcomp}
\usepackage{amssymb}
\usepackage{amsmath}
\usepackage{st}
\title{Selected extracts from the ``Comprehensive \LaTeX{} symbol
list''}
\author{}
\date{}

\begin{document}\maketitle
\section*{Introduction}
This file consists in extracts of the
\ahref{http://www.ctan.org/tex-archive/info/symbols/comprehensive/}
{Comprehensive list of \LaTeX{} symbols}.

\section{Body-text symbols}
\begin{symtable}{\latexE{} Escapable ``Special'' Characters}
\index{special characters=``special'' characters}
\index{escapable characters}
\label{special-escapable}
\begin{tabular}{*6{ll@{\qqquad}}ll}
\K\$   & \K\%   & \K\_$\,^*$  & \Kp\}   & \K\&   & \K\#   & \Kp\{   \\
\end{tabular}

\bigskip
\begin{tablenote}[*]
  The \pkgname{underscore} package redefines ``\verb+_+'' to produce
  an underscore in text mode (i.e.,~it makes it unnecessary to escape
  the underscore character).
\end{tablenote}
\end{symtable}

\begin{symtable}{Predefined \latexE{} Text-mode Commands}
\index{space, visible}
\index{inequalities}
\index{tilde}
\index{copyright}
\idxboth{legal}{symbols}
\label{text-predef}
\begin{tabular}{lll@{\qqquad}lll}
\V\textasciicircum      & \V\textless            \\
\V\textasciitilde       & \V[\ltextordfeminine]\textordfeminine   \\
\V\textasteriskcentered & \V[\ltextordmasculine]\textordmasculine \\
\V\textbackslash        & \V\textparagraph$^*$   \\
\V\textbar              & \V\textperiodcentered  \\
\V\textbraceleft$^*$    & \V\textquestiondown    \\
\V\textbraceright$^*$   & \V\textquotedblleft    \\
\V\textbullet           & \V\textquotedblright   \\
\V[\ltextcopyright]\textcopyright$^*$
                        & \V\textquoteleft       \\
\V\textdagger$^*$       & \V\textquoteright      \\
\V\textdaggerdbl$^*$    & \V[\ltextregistered]\textregistered     \\
\V\textdollar$^*$       & \V\textsection$^*$     \\
\V\textellipsis$^*$     & \V\textsterling$^*$    \\
\V\textemdash           & \V[\ltexttrademark]\texttrademark       \\
\V\textendash           & \V\textunderscore$^*$  \\
\V\textexclamdown       & \V\textvisiblespace    \\
\V\textgreater          \\
\end{tabular}

\bigskip
\twosymbolmessage

\bigskip
\usetextmathmessage[*]

\end{symtable}


\begin{symtable}{\latexE{} Commands Defined to Work in Both Math and Text Mode}
\index{dots (ellipses)} \index{ellipses (dots)}
\index{copyright}
\idxboth{legal}{symbols}
\label{math-text}
\begin{tabular}{*3{lll@{\qqquad}}lll}
\V\$ & \V\_              & \V\ddag    & \Vp\{ \\
\V\P & \V[\ltextcopyright]\copyright
                         & \V\dots    & \Vp\} \\
\V\S & \V\dag            & \V\pounds          \\
\end{tabular}

\bigskip
\twosymbolmessage
\end{symtable}
\begin{symtable}{Non-ASCII Letters (Excluding Accented Letters)}
\index{letters>non-ASCII}\index{ASCII}
\label{non-ascii}
\begin{tabular}{*4{ll@{\hspace*{3em}}}ll}
\K\aa      & \Ks\DH     & \K\L       & \K\o       & \K\ss      \\
\K\AA      & \Ks\dh     & \K\l       & \K\O       & \K\SS      \\
\K\AE      & \Ks\DJ     & \Ks\NG     & \K\OE      & \Ks\TH     \\
\K\ae      & \Ks\dj     & \Ks\ng     & \K\oe      & \Ks\th     \\
\end{tabular}

\bigskip
\begin{tablenote}[*]
  Not available in the OT1 \fntenc[OT1].  Use the \pkgname{fontenc}
  package to select an alternate \fntenc[T1], such as T1.
\end{tablenote}
\end{symtable}


\begin{symtable}{Punctuation Marks Not Found in OT1}
\index{punctuation}
\label{punc-no-OT1}
\begin{tabular}{*8l}
\Kt\guillemotleft  & \Kt\guilsinglleft & \Kt\quotedblbase & \Kt\textquotedbl \\
\Kt\guillemotright & \Kt\guilsinglright & \Kt\quotesinglbase \\
\end{tabular}
\bigskip
\begin{tablenote}
  To get these symbols, use the \pkgname{fontenc} package to select an
  alternate \fntenc[T1], such as~T1.
\end{tablenote}
\end{symtable}
\begin{symtable}{\TC\ Diacritics}
\index{accents}
\label{tc-accent-chars}
\begin{tabular}{*3{ll}}
\K\textacutedbl      & \K\textasciicaron    & \K\textasciimacron \\
\K\textasciiacute    & \K\textasciidieresis & \K\textgravedbl    \\
\K\textasciibreve    & \K\textasciigrave                         \\
\end{tabular}

\bigskip

\begin{tablenote}
  The \TC\ package defines all of the above as ordinary characters,
  not as accents.
\end{tablenote}
\end{symtable}


\begin{symtable}{\TC\ Currency Symbols}
\idxboth{currency}{symbols}
\idxboth{monetary}{symbols}
\label{tc-currency}
\begin{tabular}{*4{ll}}
\K\textbaht          & \K\textdollar$^*$     & \K\textguarani  & \K\textwon \\
\K\textcent          & \NK\textdollaroldstyle & \K\textlira     & \K\textyen \\
\NK\textcentoldstyle  & \K\textdong           & \K\textnaira    \\
\K\textcolonmonetary & \K\texteuro           & \K\textpeso     \\
\K\textcurrency      & \K\textflorin         & \K\textsterling$^*$ \\
\end{tabular}

\bigskip
\usetextmathmessage[*]

\end{symtable}

\begin{symtable}{\TC\ Legal Symbols}
\index{copyright}
\idxboth{legal}{symbols}
\label{tc-legal}
\begin{tabular}{*2{lll@{\qquad}}lll}
\V\textcircledP & \V[\ltextcopyright]\textcopyright   & \V\textservicemark \\
\NV\textcopyleft & \V[\ltextregistered]\textregistered & \V[\ltexttrademark]\texttrademark \\
\end{tabular}

\bigskip
\twosymbolmessage
\medskip
\begin{tablenote}
  \hspace*{15pt}%
  See \url{http://www.tex.ac.uk/cgi-bin/texfaq2html?label=tradesyms}
  for solutions to common problems that occur when using these symbols
  (e.g.,~getting a~``\textcircled{r}'' when you expected to get
  a~``\textregistered'').
\end{tablenote}
\end{symtable}
\begin{symtable}{Miscellaneous \TC\ Symbols}
\idxboth{musical}{symbols}
\index{tilde}
\label{tc-misc}
\begin{tabular}{lll@{\qquad}lll}
\V\textasteriskcentered & \V[\ltextordfeminine]\textordfeminine   \\
\V\textbardbl           & \V[\ltextordmasculine]\textordmasculine \\
\V\textbigcircle        & \V\textparagraph$^*$                    \\
\NV\textblank            & \V\textperiodcentered                   \\
\V\textbrokenbar        & \V\textpertenthousand                   \\
\V\textbullet           & \V\textperthousand                      \\
\V\textdagger$^*$       & \V\textpilcrow                          \\
\V\textdaggerdbl$^*$    & \V\textquotesingle                      \\
\V\textdblhyphen        & \NV\textquotestraightbase                \\
\V\textdblhyphenchar    & \NV\textquotestraightdblbase             \\
\V\textdiscount         & \V\textrecipe                           \\
\V\textestimated        & \V\textreferencemark                    \\
\V\textinterrobang      & \V\textsection$^*$                      \\
\NV\textinterrobangdown  & \NV\textthreequartersemdash              \\
\V\textmusicalnote      & \V\texttildelow                         \\
\V\textnumero           & \NV\texttwelveudash                      \\
\V\textopenbullet                                                 \\
\end{tabular}

\bigskip
\twosymbolmessage

\bigskip
\usetextmathmessage[*]

\end{symtable}



\begin{symtable}{Text-mode Accents}
\index{accents}
\label{text-accents}
\begin{tabular}{*3{ll@{\hspace*{3em}}}ll}
 \Q\"                                & \Q\`         & \Q\d         & \Q\r        \\
\Q\'                                & \QivBAR\ddag & \Qiv\G\ddag  & \NQ\t        \\
\Q\.                                & \Q\~         & \Qv\h\S      & \Q\u        \\
\Q\= & \Q\b         & \QQ{\H}{O}{o}         & \Qiv\U\ddag \\
\Q\^                                & \QQ{\c}Cc         & \Qt\k$^\dag$ & \Q\v        \\
\end{tabular}
\par\medskip
\begin{tabular}{ll@{\hspace*{3em}}ll}
\NQ\newtie$^*$ & \Qc\textcircled
\end{tabular}

\bigskip
\begin{tablenote}[*]
  Requires the \TC\ package.
\end{tablenote}

\medskip
\begin{tablenote}[\dag]
  Not available in the OT1 \fntenc[OT1].  Use the \pkgname{fontenc}
  package to select an alternate \fntenc[T1], such as T1.
\end{tablenote}

\medskip
\begin{tablenote}[\ddag]
  Requires the T4 \fntenc[T4], provided by the \FC\ package.
\end{tablenote}

\medskip
\begin{tablenote}[\S]
  Requires the T5 \fntenc[T5], provided by the \VIET\ package.
\end{tablenote}

\bigskip
\begin{tablenote}
  \index{dotless i=dotless $i~(\imath)$>text mode}
  \index{dotless j=dotless $j~(\jmath)$>text mode}
  Also note the existence of \cmdI{\i} and \cmdI{\j}, which produce
  dotless versions of ``i'' and ``j'' (viz., ``\i'' and ``\j'').  These
  are useful when the accent is supposed to replace the dot.  For
  example, ``\verb|na\"{\i}ve|'' produces a correct ``na\"{\i}ve'',
  while ``\verb|na\"{i}ve|'' would yield the rather odd-looking
  ``na\"{i}ve''.  (``\verb|na\"{i}ve|'' \emph{does} work in encodings
  other than OT1, however.)
\end{tablenote}
\end{symtable}

\section{Symbols for maths}

\begin{symtable}{Math-Mode Versions of Text Symbols}
\index{math-text}
\begin{tabular}{*3{ll}}
\X\mathdollar   & \X\mathparagraph & \X\mathsterling   \\
\X\mathellipsis & \X\mathsection   & \X\mathunderscore \\
\end{tabular}

\bigskip
\usetextmathmessage

\end{symtable}

\begin{symtable}{Binary Operators}
\idxboth{binary}{operators}
\index{division}
\label{bin}
\begin{tabular}{*4{ll}}
\X\amalg           & \X\cup          & \X\oplus    & \X\times           \\
\X\ast             & \X\dagger       & \X\oslash   & \X\triangleleft    \\
\X\bigcirc         & \X\ddagger      & \X\otimes   & \X\triangleright   \\
\X\bigtriangledown & \X\diamond      & \X\pm       & \X\unlhd$^*$       \\
\X\bigtriangleup   & \X\div          & \X\rhd$^*$  & \X\unrhd$^*$       \\
\X\bullet          & \X\lhd$^*$      & \X\setminus & \X\uplus           \\
\X\cap             & \X\mp           & \X\sqcap    & \X\vee             \\
\X\cdot            & \X\odot         & \X\sqcup    & \X\wedge           \\
\X\circ            & \X\ominus       & \X\star     & \X\wr              \\
\end{tabular}

\bigskip
\notpredefinedmessage
\end{symtable}

\begin{symtable}{Variable-sized Math Operators}
\idxboth{variable-sized}{symbols}
\index{integrals}
\label{op}
\renewcommand{\arraystretch}{1.75}  % Keep tall symbols from touching.
\begin{tabular}{*3{l@{$\:$}ll@{\qquad}}l@{$\:$}ll}
\R\bigcap    & \R\bigotimes & \R\bigwedge  & \R\prod      \\
\R\bigcup    & \R\bigsqcup  & \R\coprod    & \R\sum       \\
\R\bigodot   & \R\biguplus  & \R\int       \\
\R\bigoplus  & \R\bigvee    & \R\oint      \\
\end{tabular}
\end{symtable}

\begin{symtable}{Binary Relations}
\idxboth{relational}{symbols}
\index{tacks}
\label{rel}
\begin{tabular}{*4{ll}}
\X\approx   & \X\equiv    & \X\perp     & \X\smile  \\
\X\asymp    & \X\frown    & \X\prec     & \X\succ   \\
\X\bowtie   & \X\Join$^*$ & \X\preceq   & \X\succeq \\
\X\cong     & \X\mid      & \X\propto   & \X\vdash  \\
\X\dashv    & \X\models   & \X\sim                  \\
\X\doteq    & \X\parallel & \X\simeq                \\
\end{tabular}

\bigskip
\notpredefinedmessageABX
\end{symtable}

\begin{symtable}{Subset and Superset Relations}
\index{binary relations}
\index{relational symbols>binary}
\index{subsets}
\index{supersets}
\index{symbols>subset and superset}
\label{subsets}
\begin{tabular}{*3{ll}}
\X\sqsubset$^*$ & \X\sqsupseteq & \X\supset   \\
\X\sqsubseteq   & \X\subset     & \X\supseteq \\
\X\sqsupset$^*$ & \X\subseteq                 \\
\end{tabular}

\bigskip
\notpredefinedmessageABX
\end{symtable}

\begin{symtable}{Inequalities}
\index{binary relations}\index{relational symbols>binary}
\index{inequalities}
\label{inequal-rel}
\begin{tabular}{*5{ll}}
\X\geq & \X\gg & \X\leq & \X\ll & \X\neq \\
\end{tabular}
\end{symtable}

\begin{symtable}{Arrows}
\index{arrows}
\label{arrow}
\begin{tabular}{*3{ll}}
\X\Downarrow          & \X\longleftarrow      & \X\nwarrow     \\
\X\downarrow          & \X\Longleftarrow      & \X\Rightarrow  \\
\X\hookleftarrow      & \X\longleftrightarrow & \X\rightarrow  \\
\X\hookrightarrow     & \X\Longleftrightarrow & \X\searrow     \\
\X\leadsto$^*$        & \X\longmapsto         & \X\swarrow     \\
\X\leftarrow          & \X\Longrightarrow     & \X\uparrow     \\
\X\Leftarrow          & \X\longrightarrow     & \X\Uparrow     \\
\X\Leftrightarrow     & \X\mapsto             & \X\updownarrow \\
\X\leftrightarrow     & \X\nearrow$^\dag$     & \X\Updownarrow \\
\end{tabular}

\bigskip
\notpredefinedmessage

\bigskip
\begin{tablenote}[\dag]
  See the note beneath Table~\ref{extensible-accents} for information
  about how to put a diagonal arrow across a mathematical expression%
\ifhavecancel
  ~(as in ``$\cancelto{0}{\nabla \cdot \vec{B}}\quad$'')
\fi
.
\end{tablenote}
\end{symtable}


\begin{symtable}{Harpoons}
\index{harpoons}
\label{harpoons}
\begin{tabular}{*3{ll}}
\X\leftharpoondown   & \X\rightharpoondown  & \X\rightleftharpoons \\
\X\leftharpoonup     & \X\rightharpoonup                           \\
\end{tabular}
\end{symtable}

\begin{symtable}{Extension Characters}
\index{extension characters}
\label{ext}
\begin{tabular}{*2{ll}}
\X\relbar & \X\Relbar \\
\end{tabular}
\end{symtable}

\begin{symtable}{Log-like Symbols}
\idxboth{log-like}{symbols}
\index{atomic math objects}
\index{limits}
\label{log}
\begin{tabular}{*8l}
\Z\arccos & \Z\cos  & \Z\csc & \Z\exp & \Z\ker    & \Z\limsup & \Z\min & \Z\sinh \\
\Z\arcsin & \Z\cosh & \Z\deg & \Z\gcd & \Z\lg     & \Z\ln     & \Z\Pr  & \Z\sup  \\
\Z\arctan & \Z\cot  & \Z\det & \Z\hom & \Z\lim    & \Z\log    & \Z\sec & \Z\tan  \\
\Z\arg    & \Z\coth & \Z\dim & \Z\inf & \Z\liminf & \Z\max    & \Z\sin & \Z\tanh
\end{tabular}

\bigskip
\begin{tablenote}
  Calling the above ``symbols'' may be a bit
  misleading.\footnotemark{} Each log-like symbol merely produces the
  eponymous textual equivalent, but with proper surrounding spacing.
  As \cmd{\bmod} and \cmd{\pmod} are arguably not symbols we
  refer the reader to the Short Math Guide for
  \latex~\cite{Downes:smg} for samples.
\end{tablenote}
\end{symtable}
\footnotetext{Michael\index{Downes, Michael J.} J. Downes prefers the
more general term, ``atomic\index{atomic math objects} math objects''.}


\begin{symtable}{\TC\ Text-mode Arrows}
\index{arrows}
\label{tc-arrows}
\begin{tabular}{*2{ll}}
\K\textdownarrow & \K\textrightarrow \\
\K\textleftarrow & \K\textuparrow    \\
\end{tabular}
\end{symtable}

\begin{symtable}{Math-mode Accents}
\index{accents}
\index{tilde}
\label{math-accents}
\begin{tabular}{*4{ll}}
\W\acute{a}    & \W\check{a}    & \W\grave{a}    & \W\tilde{a} \\
\W\bar{a}      & \NW\ddot{a}     & \W\hat{a}      & \NW\vec{a}   \\
\W\breve{a}    & \W\dot{a}      & \W\mathring{a}               \\
\end{tabular}

\bigskip

\begin{tablenote}
  \index{dotless i=dotless $i~(\imath)$>math mode}
  \index{dotless j=dotless $j~(\jmath)$>math mode}
  Also note the existence of \cmdX{\imath} and \cmdX{\jmath}, which
  produce dotless versions of ``\textit{i}'' and ``\textit{j}''.  (See
  Table~\vref{ord}.)  These are useful when the accent is supposed to
  replace the dot.  For example, ``\verb|\hat{\imath}|'' produces a
  correct ``$\,\hat{\imath}\,$'', while ``\verb|\hat{i}|'' would yield
  the rather odd-looking ``\,$\hat{i}\,$''.
\end{tablenote}
\end{symtable}




\begin{symtable}{Greek Letters}
\index{Greek}\index{alphabets>Greek}
\label{greek}
\begin{tabular}{*8l}
\X\alpha        &\X\theta       &\X o           &\X\tau         \\
\X\beta         &\X\vartheta    &\X\pi          &\X\upsilon     \\
\X\gamma        &\X\iota        &\X\varpi       &\X\phi         \\
\X\delta        &\X\kappa       &\X\rho         &\X\varphi      \\
\X\epsilon      &\X\lambda      &\X\varrho      &\X\chi         \\
\X\varepsilon   &\X\mu          &\X\sigma       &\X\psi         \\
\X\zeta         &\X\nu          &\X\varsigma    &\X\omega       \\
\X\eta          &\X\xi                                          \\
                                                                \\
\X\Gamma        &\X\Lambda      &\X\Sigma       &\X\Psi         \\
\X\Delta        &\X\Xi          &\X\Upsilon     &\X\Omega       \\
\X\Theta        &\X\Pi          &\X\Phi
\end{tabular}

\bigskip
\begin{tablenote}
  The remaining Greek majuscules\index{majuscules} can be produced
  with ordinary Latin letters.  The symbol ``M'', for instance, is
  used for both an uppercase ``m'' and an uppercase ``$\mu$''.
\end{tablenote}
\end{symtable}

\begin{symtable}{Letter-like Symbols}
\idxboth{letter-like}{symbols}
\index{tacks}
\label{letter-like}
\begin{tabular}{*5{ll}}
\X\bot    & \X\forall & \X\imath & \X\ni      & \X\top \\
\X\ell    & \X\hbar   & \X\in    & \X\partial & \X\wp  \\
\X\exists & \X\Im     & \X\jmath & \X\Re               \\
\end{tabular}
\end{symtable}

\begin{symtable}{Variable-sized Delimiters}
\index{delimiters}
\index{delimiters>variable-sized}
\label{dels}
\renewcommand{\arraystretch}{1.75}  % Keep tall symbols from touching.
\begin{tabular}{lll@{\qquad}lll@{\hspace*{1.5cm}}lll@{\qquad}lll}
\N\downarrow & \N\Downarrow & \N{[}           & \N[\magicrbrack]{]} \\
\N\langle    & \N\rangle    & \N|$^*$
                                              & \N\| \\
\N\lceil     & \N\rceil     & \N\uparrow      & \N\Uparrow          \\
\N\lfloor    & \N\rfloor    & \N\updownarrow  & \N\Updownarrow      \\
\N(          & \N)          & \N\{           & \N\}               \\
\N/          & \N\backslash                                         \\
\end{tabular}

\bigskip
\begin{tablenote}
  When used with \cmd{\left} and \cmd{\right}, these symbols expand to
  the height of the enclosed math expression.  Note that \cmdX{\vert}
  is a synonym for \verb+|+, and \cmdX{\Vert} is a synonym for
  \verb+\|+.
\end{tablenote}

\bigskip
\begin{tablenote}[*]
  $\varepsilon$-\TeX{}\index{e-tex=$\varepsilon$-\TeX} provides a
  \cmd{\middle} analogue to \cmd{\left} and \cmd{\right} that can be
  used to make an internal ``$|$'' (often used to indicate
  ``evaluated\index{evaluated at=evaluated at ($\vert$)} at'') expand
  to the height of the surrounding \cmd{\left} and \cmd{\right}
  symbols.  A similar effect can be achieved in conventional \latex
  using the \pkgname{braket} package.
\end{tablenote}
\end{symtable}

\begin{symtable}{Large, Variable-sized Delimiters}
\index{delimiters}
\index{delimiters>variable-sized}
\label{ldels}
\renewcommand{\arraystretch}{2.5}  % Keep tall symbols from touching.
\begin{tabular}{*3{lll@{\qquad}}lll}
\Y\lmoustache & \Y\rmoustache & \Y\lgroup    & \Y\rgroup \\
\Y\arrowvert  & \Y\Arrowvert  & \Y\bracevert
\end{tabular}

\bigskip
\begin{tablenote}
  These symbols \emph{must} be used with \cmd{\left} and \cmd{\right}.
  The \ABX\ package, however, redefines
  \cmdI[$\string\big\string\lgroup$]{\lgroup} and
  \cmdI[$\string\big\string\rgroup$]{\rgroup} so that those symbols
  can work without \cmd{\left} and \cmd{\right}.
\end{tablenote}
\end{symtable}

\begin{symtable}{\TC\ Text-mode Delimiters}
\index{delimiters}
\index{delimiters>text-mode}
\label{tc-delimiters}
\begin{tabular}{*2{ll}}
\K\textlangle    & \K\textrangle    \\
\K\textlbrackdbl & \K\textrbrackdbl \\
\NK\textlquill    & \NK\textrquill    \\
\end{tabular}
\end{symtable}

\begin{symtable}{Extensible Accents}
\index{accents}
\idxboth{extensible}{accents}
\idxboth{extensible}{arrows}
\index{tilde}
\index{tilde>extensible}
\index{extensible tildes}
\label{extensible-accents}
\renewcommand{\arraystretch}{1.5}
\begin{tabular}{*4l}
\NW\widetilde{abc}$^*$         & \NW\widehat{abc}$^*$    \\
\WD\overleftarrow{abc}$^\dag$  & \WD\overrightarrow{abc}$^\dag$ \\
\WD\overline{abc}              & \WD\underline{abc}      \\
\WD\overbrace{abc}             & \WD\underbrace{abc}     \\[5pt]
\WD\sqrt{abc}$^\ddag$                                   \\
\end{tabular}

\bigskip

\begin{tablenote}
  \def\longdivsign{%
    \ensuremath{\overline{\vphantom{)}%
      \hbox{\smash{\raise3.5\fontdimen8\textfont3\hbox{$)$}}}%
      abc}}}

  \index{long division|(}
  \index{division|(}
  \index{polynomial division|(}

  As demonstrated in a 1997 TUGboat\index{TUGboat} article about
  typesetting long-division problems~\cite{Gibbons:longdiv}, an
  extensible long-division sign (``\,\longdivsign\,'') can be faked by
  putting a ``\verb|\big)|'' in a \texttt{tabular} environment with an
  \verb|\hline| or \verb|\cline| in the preceding row.  The article
  also presents a piece of code (uploaded to CTAN\idxCTAN{} as
  \texttt{longdiv.tex}%
  \index{longdiv=\textsf{longdiv} (package)}%
  \index{packages>\textsf{longdiv}}) that automatically solves and
  typesets---by putting an \cmdW{\overline} atop ``\verb|\big)|'' and
  the desired text---long-division problems.  See also the
  \pkgname{polynom} package, which automatically solves and typesets
  polynomial-division problems in a similar manner.

  \index{long division|)}
  \index{division|)}
  \index{polynomial division|)}
\end{tablenote}

\begin{tablenote}[\dag]
  If you're looking for an extensible \emph{diagonal} line or arrow to
  be used for canceling or reducing mathematical
  subexpressions\index{arrows>diagonal, for reducing subexpressions}
\ifhavecancel
  (e.g.,~``$\cancel{x + -x}$'' or ``$\cancelto{5}{3+2}\quad$'')
\fi
  then consider using the \pkgname{cancel} package.
\end{tablenote}

\bigskip

\begin{tablenote}[\ddag]
  With an optional argument, \verb|\sqrt| typesets nth roots.  For
  example, ``\verb|\sqrt[3]{abc}|'' produces~``$\!\sqrt[3]{abc}$\,''
  and ``\verb|\sqrt[n]{abc}|'' produces~``$\!\sqrt[n]{abc}$\,''.
\end{tablenote}
\end{symtable}
\begin{symtable}{Dots}
\idxboth{dot}{symbols}
\index{dots (ellipses)} \index{ellipses (dots)}
\label{dots}
\ifMDOTS
  \def\MDfn{$^\dag$}%
\else
  \def\MDfn{}%
\fi    % MDOTS test
\begin{tabular}{*{3}{ll@{\hspace*{1.5cm}}}ll}
\X\cdotp & \X\colon$^*$  & \X\ldotp & \X\vdots\MDfn \\
\X\cdots & \X\ddots\MDfn & \X\ldots                 \\
\end{tabular}

\bigskip

\begin{tablenote}[*]
  While ``\texttt{:}'' is valid in math mode, \cmd{\colon} uses
  different surrounding spacing. 
\end{tablenote}

\ifMDOTS
\bigskip

\begin{tablenote}[\dag]
  The \MDOTS\ package redefines \cmdX{\ddots} and \cmdX{\vdots} to
  make them scale properly with font size.  (They normally scale
  horizontally but not vertically.)  \cmdX{\fixedddots} and
  \cmdX{\fixedvdots} provide the original, fixed-height functionality
  of \latexE's \cmdX{\ddots} and \cmdX{\vdots} macros.
\end{tablenote}

\fi    % MDOTS test
\end{symtable}
\begin{symtable}{Miscellaneous \TC\ Text-mode Math Symbols}
\index{fractions}
\label{tc-math}
\ifFRAC
  \def\FRACfn{$^\dag$}
\else
  \def\FRACfn{}
\fi
\begin{tabular}{*3{ll}}
\K\textdegree$^*$      & \K\textonehalf\FRACfn    & \K\textthreequarters\FRACfn \\
\K\textdiv             & \K\textonequarter\FRACfn & \K\textthreesuperior \\
\K\textfractionsolidus & \K\textonesuperior       & \K\texttimes         \\
\K\textlnot            & \K\textpm                & \K\texttwosuperior   \\
\K\textminus           & \K\textsurd                                     \\
\end{tabular}

\bigskip

\begin{tablenote}[*]
  If you prefer a larger degree symbol you might consider defining one
  as ``\verb|\ensuremath{^\circ}|''~(``$^\circ$'')%
  \indexcommand[$\string\circ$]{\circ}.
\end{tablenote}

\ifFRAC
  \bigskip
  \begin{tablenote}[\dag]
    \pkgname{nicefrac} (part of the \pkgname{units} package) can be
    used to construct vulgar fractions like ``\nicefrac{1}{2}'',
    ``\nicefrac{1}{4}'', ``\nicefrac{3}{4}'', and even
    ``\nicefrac{c}{o}''\index{care of=care of (\nicefrac{c}{o})}.
  \end{tablenote}
\fi    % FRAC test
\end{symtable}

\begin{symtable}{\TC\ Text-mode Science and Engineering Symbols}
\label{tc-science}
\begin{tabular}{*4{ll}}
\K\textcelsius & \K\textmho & \K\textmu & \K\textohm \\
\end{tabular}
\end{symtable}

\begin{symtable}{\TC\ Genealogical Symbols}
\idxboth{genealogical}{symbols}
\label{genealogical}
\begin{tabular}{*3{ll}}
\K\textborn     & \K\textdivorced & \K\textmarried  \\
\K\textdied     & \NK\textleaf     \\
\end{tabular}
\end{symtable}


\begin{symtable}{Miscellaneous \latexE{} Math Symbols}
\idxboth{miscellaneous}{symbols}
\index{card suits}
\index{diamonds (suit)}
\index{hearts (suit)}
\index{clubs (suit)}
\index{spades (suit)}
\idxboth{musical}{symbols}
\index{dots (ellipses)}
\index{ellipses (dots)}
\index{null set}
\index{dotless i=dotless $i~(\imath)$>math mode}
\index{dotless j=dotless $j~(\jmath)$>math mode}
\index{angles}
\label{ord}
\ifAMS
  \def\AMSfn{$^\ddag$}
\else
  \def\AMSfn{}
\fi
\begin{tabular}{*4{ll}}
\X\aleph          & \X\Diamond$^*$    & \X\infty   & \X\prime     \\
\X\angle          & \X\diamondsuit    & \X\mho$^*$ & \X\sharp     \\
\X\backslash      & \X\emptyset\AMSfn & \X\nabla   & \X\spadesuit \\
\X\Box$^{*,\dag}$ & \X\flat           & \X\natural & \X\surd      \\
\X\clubsuit       & \X\heartsuit      & \X\neg     & \X\triangle  \\
\end{tabular}

\bigskip
\notpredefinedmessage

\bigskip
\begin{tablenote}[\dag]
  To use \cmdX{\Box}---or any other symbol---as an end-of-proof
  (Q.E.D\@.)\index{Q.E.D.}\index{end of proof}\index{proof, end of}
  marker, consider using the \pkgname{ntheorem} package, which
  properly juxtaposes a symbol with the end of the proof text.
\end{tablenote}

\ifAMS
  \bigskip
  \begin{tablenote}[\ddag]
    Many people prefer the look of \AMS's \cmdX{\varnothing}
    (Table~\ref{ams-misc}) to that of \latex's \cmdX{\emptyset}.
  \end{tablenote}
\fi    % AMS test

\end{symtable}

\section*{AMS symbols}


\begin{symtable}[AMS]{\AMS\ Commands Defined to Work in Both Math and Text Mode}
\label{ams-math-text}
\begin{tabular}{*2{ll@{\qquad}}ll}
\X\checkmark & \X\circledR & \X\maltese
\end{tabular}
\end{symtable}

\begin{symtable}[AMS]{\AMS\ Binary Operators}
\idxboth{binary}{operators}
\index{semidirect products}
\label{ams-bin}
\begin{tabular}{*3{ll}}
\X\barwedge        & \X\circledcirc     & \X\intercal        \\
\X\boxdot          & \X\circleddash     & \X\leftthreetimes  \\
\X\boxminus        & \X\Cup             & \X\ltimes          \\
\X\boxplus         & \X\curlyvee        & \X\rightthreetimes \\
\X\boxtimes        & \X\curlywedge      & \X\rtimes          \\
\X\Cap             & \X\divideontimes   & \X\smallsetminus   \\
\X\centerdot       & \X\dotplus         & \X\veebar          \\
\X\circledast      & \X\doublebarwedge  \\
\end{tabular}
\end{symtable}

\begin{symtable}[AMS]{\AMS\ Variable-sized Math Operators}
\idxboth{variable-sized}{symbols}
\index{integrals}
\label{ams-large}
\renewcommand{\arraystretch}{2.5}  % Keep tall symbols from touching.
\begin{tabular}{l@{$\:$}ll@{\qquad}l@{$\:$}ll}
\R[\AMSiint]\iint     & \R[\AMSiiint]\iiint       \\
\R[\AMSiiiint]\iiiint & \R[\AMSidotsint]\idotsint \\
\end{tabular}
\end{symtable}

\begin{symtable}[AMS]{\AMS\ Binary Relations}
\index{binary relations}
\index{relational symbols>binary}
\label{ams-rel}
\begin{tabular}{*3{ll}}
\X\approxeq      & \X\eqcirc        & \X\succapprox    \\
\X\backepsilon   & \X\fallingdotseq & \X\succcurlyeq   \\
\X\backsim       & \X\multimap      & \X\succsim       \\
\X\backsimeq     & \X\pitchfork     & \X\therefore     \\
\X\because       & \X\precapprox    & \NX\thickapprox   \\
\X\between       & \X\preccurlyeq   & \NX\thicksim      \\
\X\Bumpeq        & \X\precsim       & \X\varpropto     \\
\X\bumpeq        & \X\risingdotseq  & \X\Vdash         \\
\X\circeq        & \NX\shortmid      & \X\vDash         \\
\X\curlyeqprec   & \NX\shortparallel & \X\Vvdash        \\
\X\curlyeqsucc   & \NX\smallfrown    &                  \\
\X\doteqdot      & \NX\smallsmile    &                  \\
\end{tabular}
\end{symtable}

\begin{symtable}[AMS]{\AMS\ Negated Binary Relations}
\index{binary relations>negated}
\index{relational symbols>negated binary}
\label{ams-nrel}
\begin{tabular}{*3{ll}}
\X\ncong     & \NX\nshortparallel & \X\nVDash      \\
\X\nmid      & \X\nsim           & \X\precnapprox \\
\X\nparallel & \X\nsucc          & \X\precnsim    \\
\X\nprec     & \X\nsucceq        & \X\succnapprox \\
\X\npreceq   & \X\nvDash         & \X\succnsim    \\
\NX\nshortmid & \X\nvdash                          \\
\end{tabular}
\end{symtable}
\begin{symtable}[AMS]{\AMS\ Subset and Superset Relations}
\index{binary relations}
\index{relational symbols>binary}
\index{subsets}
\index{supersets}
\index{symbols>subset and superset}
\label{ams-subsets}
\begin{tabular}{*3{ll}}
\X\nsubseteq  & \X\subseteqq  & \X\supsetneqq    \\
\X\nsupseteq  & \X\subsetneq  & \NX\varsubsetneq  \\
\NX\nsupseteqq & \X\subsetneqq & \NX\varsubsetneqq \\
\X\sqsubset   & \X\Supset     & \NX\varsupsetneq  \\
\X\sqsupset   & \X\supseteqq  & \NX\varsupsetneqq \\
\X\Subset     & \X\supsetneq                     \\
\end{tabular}
\end{symtable}


\begin{symtable}[AMS]{\AMS\ Inequalities}
\index{binary relations}\index{relational symbols>binary}
\index{inequalities}
\label{ams-inequal-rel}
\renewcommand{\arraystretch}{1.5}   % Keep visually similar symbols from touching.
\begin{tabular}{*4{ll}}
\X\eqslantgtr  & \X\gtrdot      & \X\lesseqgtr   & \X\ngeq        \\
\X\eqslantless & \X\gtreqless   & \X\lesseqqgtr  & \NX\ngeqq       \\
\X\geqq        & \X\gtreqqless  & \X\lessgtr     & \NX\ngeqslant   \\
\X\geqslant    & \X\gtrless     & \X\lesssim     & \X\ngtr        \\
\X\ggg         & \X\gtrsim      & \X\lll         & \X\nleq        \\
\X\gnapprox    & \NX\gvertneqq   & \X\lnapprox    & \NX\nleqq       \\
\X\gneq        & \X\leqq        & \X\lneq        & \NX\nleqslant   \\
\X\gneqq       & \X\leqslant    & \X\lneqq       & \X\nless       \\
\X\gnsim       & \X\lessapprox  & \X\lnsim       &                \\
\X\gtrapprox   & \X\lessdot     & \NX\lvertneqq   &                \\
\end{tabular}
\end{symtable}

\begin{symtable}[AMS]{\AMS\ Triangle Relations}
\index{triangle relations}\index{relational symbols>triangle}
\label{ams-triangle-rel}
\begin{tabular}{*4{ll}}
\X\blacktriangleleft  & \X\ntrianglelefteq  & \X\trianglelefteq  & \X\vartriangleleft  \\
\X\blacktriangleright & \X\ntriangleright   & \X\triangleq       & \X\vartriangleright \\
\X\ntriangleleft      & \X\ntrianglerighteq & \X\trianglerighteq                       \\
\end{tabular}
\end{symtable}

\begin{symtable}[AMS]{\AMS\ Arrows}
\index{arrows}
\label{ams-arrows}
\begin{tabular}{*3{ll}}
\X\circlearrowleft  & \X\leftleftarrows      & \X\rightleftarrows   \\
\X\circlearrowright & \X\leftrightarrows     & \X\rightrightarrows  \\
\X\curvearrowleft   & \X\leftrightsquigarrow & \X\rightsquigarrow   \\
\X\curvearrowright  & \X\Lleftarrow          & \X\Rsh               \\
\X\dashleftarrow    & \X\looparrowleft       & \X\twoheadleftarrow  \\
\X\dashrightarrow   & \X\looparrowright      & \X\twoheadrightarrow \\
\X\downdownarrows   & \X\Lsh                 & \X\upuparrows        \\
\X\leftarrowtail    & \X\rightarrowtail      &                      \\
\end{tabular}
\end{symtable}


\begin{symtable}[AMS]{\AMS\ Negated Arrows}
\index{arrows>negated}
\label{ams-narrows}
\begin{tabular}{*3{ll}}
\X\nLeftarrow      & \X\nLeftrightarrow & \X\nRightarrow     \\
\X\nleftarrow      & \X\nleftrightarrow & \X\nrightarrow     \\
\end{tabular}
\end{symtable}

\begin{symtable}[AMS]{\AMS\ Harpoons}
\index{harpoons}
\label{ams-harpoons}
\begin{tabular}{*3{ll}}
\X\downharpoonleft  & \X\leftrightharpoons                        & \X\upharpoonleft  \\
\X\downharpoonright & \X\rightleftharpoons & \X\upharpoonright \\
\end{tabular}
\end{symtable}
\begin{symtable}[AMS]{\AMS\ Log-like Symbols}
\idxboth{log-like}{symbols}
\index{atomic math objects}
\index{limits}
\label{ams-log}
\renewcommand{\arraystretch}{1.5}  % Keep tall symbols from touching.
\begin{tabular}{*2{ll@{\qquad}}ll}
\X\injlim     & \NX\varinjlim  & \X\varlimsup  \\
\X\projlim    & \X\varliminf  & \NX\varprojlim
\end{tabular}


\bigskip
\begin{tablenote}
  Load the \pkgname{amsmath} package to get these symbols.
  As \cmd{\mod} and \cmd{\pod} are arguably not
  symbols we refer the reader to the Short Math Guide for
  \latex~\cite{Downes:smg} for samples.
\end{tablenote}
\end{symtable}

\begin{symtable}[AMS]{\AMS\ Greek Letters}
\index{Greek}\index{alphabets>Greek}
\label{ams-greek}
\begin{tabular}{*4l}
\X\digamma      &\X\varkappa
\end{tabular}
\end{symtable}


\begin{symtable}[AMS]{\AMS\ Hebrew Letters}
\index{Hebrew}\index{alphabets>Hebrew}
\label{ams-hebrew}
\begin{tabular}{*6l}
\X\beth & \X\gimel & \X\daleth
\end{tabular}

\bigskip
\begin{tablenote}
\cmdX{\aleph} appears in Table~\vref{ord}.
\end{tablenote}
\end{symtable}
\begin{symtable}[AMS]{\AMS\ Letter-like Symbols}
\idxboth{letter-like}{symbols}
\label{ams-letter-like}
\begin{tabular}{*3{ll}}
\X\Bbbk       & \X\complement & \X\hbar    \\
\X\circledR   & \X\Finv       & \X\hslash  \\
\X\circledS   & \X\Game       & \X\nexists \\
\end{tabular}
\end{symtable}


\begin{symtable}[AMS]{\AMS\ Delimiters}
\index{delimiters}
\label{ams-del}
\begin{tabular}{*2{ll}}
\X\ulcorner & \X\urcorner \\
\X\llcorner & \X\lrcorner
\end{tabular}
\end{symtable}

\begin{symtable}[AMS]{\AMS\ Variable-sized Delimiters}
\index{delimiters}
\index{delimiters>variable-sized}
\label{ams-var-del}
\renewcommand{\arraystretch}{2.5}  % Keep tall symbols from touching.
\begin{tabular}{lll@{\qquad}lll}
\N\lvert & \N\rvert \\
\N\lVert & \N\rVert \\
\end{tabular}
\bigskip
\begin{tablenote}
  According to the \texttt{amsmath}
  documentation~\cite{AMS1999:amsmath}, the preceding symbols are
  intended to be used as delimiters (e.g.,~as in ``$\lvert -z
  \rvert$'') while the \cmdX{\vert} and \cmdX{\Vert} symbols
  (Table~\vref{dels}) are intended to be used as operators (e.g.,~as
  in ``$p \vert q$'').
\end{tablenote}
\end{symtable}

\begin{symtable}[AMS]{\AMS\ Math-mode Accents}
\index{accents}
\label{ams-math-accents}
\begin{tabular}{ll@{\hspace*{2em}}ll}
\NW\dddot{a}    & \NW\ddddot{a} \\
\end{tabular}

\bigskip

\begin{tablenote}
  These accents are also provided by the \ABX\ package.
\end{tablenote}
\end{symtable}

\begin{symtable}[AMS]{\AMS\ Extensible Accents}
\idxboth{extensible}{accents}
\idxboth{extensible}{arrows}
\label{extensible-arrows}
\renewcommand{\arraystretch}{1.5}
\begin{tabular}{ll@{\qquad}ll}
\W\overleftrightarrow{abc}  & \W\underleftrightarrow{abc} \\
\W\underleftarrow{abc}      & \W\underrightarrow{abc}     \\[2ex]
\multicolumn{4}{p{0.75\textwidth}}{%
  The following are a sort of ``reverse accent'' in that the argument
  text serves as a superscript to the arrow.  In addition, the
  optional first argument (not shown) serves as a subscript to the
  arrow.  See the Short Math Guide for \latex~\cite{Downes:smg} for
  further examples.
} \\~\\[-2ex]
\W\xleftarrow{abc}          & \W\xrightarrow{abc}         \\
\end{tabular}
\end{symtable}


\begin{symtable}[AMS]{\AMS\ Dots}
\idxboth{dot}{symbols}
\index{dots (ellipses)} \index{ellipses (dots)}
\label{ams-dots}
\begin{tabular}{*{2}{ll@{\hspace*{1.5cm}}}ll}
\X[\cdots]\dotsb & \X[\cdots]\dotsi & \X[\ldots]\dotso \\
\X[\ldots]\dotsc & \X[\cdots]\dotsm                    \\
\end{tabular}

\bigskip
\begin{tablenote}
  The \AMS\ dot symbols are named according to their intended usage:
  \cmdI[$\string\cdots$]{\dotsb} between pairs of binary operators/relations,
  \cmdI[$\string\ldots$]{\dotsc} between pairs of commas,
  \cmdI[$\string\cdots$]{\dotsi} between pairs of integrals,
  \cmdI[$\string\cdots$]{\dotsm} between pairs of multiplication signs, and
  \cmdI[$\string\ldots$]{\dotso} between other symbol pairs.
\end{tablenote}
\end{symtable}

\begin{symtable}[AMS]{Miscellaneous \AMS\ Math Symbols}
\idxboth{miscellaneous}{symbols}
\index{stars}
\index{triangles}
\index{null set}
\index{angles}
\label{ams-misc}
\begin{tabular}{*3{ll}}
\X\angle & \X\blacktriangledown & \X\mho            \\
\X\backprime        & \X\diagdown          & \X\sphericalangle \\
\X\bigstar          & \X\diagup            & \X\square         \\
\X\blacklozenge     & \X\eth               & \X\triangledown   \\
\X\blacksquare      & \X\lozenge           & \X\varnothing     \\
\X\blacktriangle    & \X\measuredangle     & \X\vartriangle    \\

\end{tabular}
\end{symtable}

\end{document}
