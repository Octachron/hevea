\documentstyle[fullpage,isolatin1,hevea]{article}

\addtolength{\topmargin}{-1cm}
\makeatletter

\typeout{``page''}

% cedric: removed for llncs

%page layout
%\oddsidemargin 0cm
%\evensidemargin 0cm
%\topmargin 10pt

%\headheight 0pt	
%\headsep 0pt	
%\footheight 0pt
%\footskip 24pt	
%\textheight 24cm
%\textwidth 15.5cm
%\footnotesep 1cm

%\baselineskip = 24pt plus 4pt minus 3pt
%\voffset=-0.5cm
%\hoffset=0.5cm

\makeatother

\makeatletter

% on  active  et  desactive  le  mode  frtypo  par \frtypotrue et frtypofalse
% quelquesoit le mode les caract\`eres :  ; ?   !   sont actifs. 
% En  particulier  ces  caract\`eres  ne  peuvent  plus \^etre  mis  dans des
% \'etiquettes.

\def\halfspace{\kern .166em}

\def\space@pv{\ifmmode ;\else \halfspace;\fi}
\def\space@dc{\ifmmode :\else \halfspace:\fi}
\def\space@pi{\ifmmode ?\else \halfspace?\fi}
\def\space@pe{\ifmmode !\else \halfspace!\fi}

\let\nospace@pv=;
\let\nospace@dc=:
\let\nospace@pi=?
\let\nospace@pe=!

\catcode`;=\active \catcode`:=\active \catcode`?=\active \catcode`!=\active

\def \frtypo
    {\let;=\space@pv\let:=\space@dc\let!=\space@pe\let?=\space@pi}
\def \entypo 
    {\let;=\nospace@pv\let:=\nospace@dc\let!=\nospace@pe\let?=\nospace@pi}

\frtypo

\def\@sanitize
 {\@makeother\ \@makeother\\\@makeother\$\@makeother\&%
\@makeother\#\@makeother\^\@makeother\^^K\@makeother\_\@makeother\^^A%
\@makeother\%\@makeother\~\@makeother\;\@makeother\?\@makeother%
\:\@makeother\!}

\let \oldspecials \dospecials 
\def \dospecials {\oldspecials \do\:\do\!\do \?\do\;}

%\scrollmode

\makeatletter
%===========================================================================
% Personals macros (They might have strange latex side effects
%-----------------

\def\noparindent{\everypar={\hskip -\parindent\everypar={}}}
\def \para {\medskip\noparindent}
\def \pari {\medskip\noparindent}
\def \parii {\smallskip\noparindent}
\let\paro=\noparindent


%===========================================================================
% Definitions Theorems 
%---------------------

\newif \iffrench 
\if \csname french\endcsname \relax  \frenchfalse \else  \frenchtrue\fi

\if \csname definition\endcsname\relax
\newtheorem{definition}{D\iffrench \'e\else e\fi  finition}\fi
\def\bdf{\begin{definition}\begin{em}}
\def\edf{\ifvmode\vskip-\lastskip\nopagebreak\else
         \penalty 100\fi\end{em}\end{definition}}

\if \csname theorem\endcsname\relax
\newtheorem{theorem}{\iffrench Th�or�me\else Theorem\fi}\fi
\def\bthm{\begin{theorem}}
\def\ethm{\end{theorem}}

\newtheorem{algorithm}{Algorithm\if\french e\else\fi}
\def\balgo{\begin{algorithm}}
\def\ealgo{\end{algorithm}}

\if \csname proposition\endcsname\relax
\newtheorem{proposition}{Proposition}\fi
\def\bprop{\begin{proposition}}
\def\eprop{\end{proposition}}

\if \csname lemma\endcsname\relax 
\newtheorem{lemma}[proposition]{Lemm\iffrench e\else a\fi}\fi
\def\blem{\begin{lemma}}
\def\elem{\end{lemma}}

\if \csname corollary\endcsname\relax
\newtheorem{corollary}[proposition]{Corollary}\fi
\def\bcor{\begin{corollary}}
\def\ecor{\end{corollary}}

\def\brmk{\trivlist \item[\hskip \labelsep {\bf {Note}}]}
\def\ermk{\endtrivlist}

\newtheorem{Note}{Note}
\def\bnote{\begin{note}\begin{em}}
\def\enote{\end{em}\end{note}}

\if \csname example\endcsname\relax
%\newtheorem{example}[remark]{Ex\iffrench a\else e\fi mple}\fi
\newtheorem{example}{Ex\iffrench a\else e\fi mple}\fi
\def\bxple{\trivlist \item[\hskip \labelsep {\bf {Example}}]}
%{\begin{example}\begin{em}}
\def\exple{\endtrivlist}
%{\end{em}\end{example}}

\def\@begintheorem#1#2{\it \trivlist \item[\hskip \labelsep{\bf #1\ #2}]}
\def\@opargbegintheorem#1#2#3{\it \trivlist
      \item[\hskip \labelsep{\bf #1\ #2\ (#3)}]}
\def\@endtheorem{\endtrivlist}

\newcount \indemo
\def \makeno #1{\global \advance \indemo by 1\global
      \expandafter \edef \csname demo/\the\cdemo/#1\endcsname{\the\indemo}}
\def \refno #1{\if \relax\csname demo/\the\cdemo/#1\endcsname \relax
  \write0 {Demo counter #1 undefined}??\else
   \csname demo/\the\cdemo/#1\endcsname\fi}
\def \labno #1{\makeno {#1}\refno {#1}}

\newif \ifdemo \demotrue
\newcount \cdemo
\newenvironment{demo}{\bdemo}{\edemo}
\def\bdemotrue{\trivlist \item[\hskip \labelsep
  {\underline
        {\iffrench Preuve\else Proof\fi}:}]\advance\cdemo by 1\indemo 0}
\def\edemo{\penalty 100\hfill\rule{2mm}{2mm} \endtrivlist\@doendpe}
\long \def \bdemofalse #1\edemo{\relax}
\outer \def \bdemo 
  {\ifdemo \bdemotrue \else \expandafter \bdemofalse \fi\indent}

\newif \ifhint \hintfalse
\newtheorem{hint}{\rm \underline {Hint}}
\def\bhintiftrue{\trivlist \item[\hskip \labelsep {\underline {Hint}:}]}
\def\ehint{\penalty 100\hfill\noindent\rule{2mm}{2mm}\endtrivlist}
\long \def \bhintiffalse #1\ehint{\relax}
\outer \def \bhint
  {\ifhint \bhintiftrue \else \expandafter \bhintiffalse \fi}

\newif \ifexpert \expertfalse
\newtheorem{expert}{Remark}
\def\bexpertiftrue{\trivlist \item[\hskip \labelsep {\underline {Hint}:}]}
\def\eexpert{\penalty 100\hfill\noindent\rule{2mm}{2mm}\endtrivlist}
\long \def \bexpertiffalse #1\eexpert{\relax}
\outer \def \bexpert 
  {\ifexpert \bexpertiftrue \else  \expandafter \bexpertiffalse \fi}



\def\notitle{\hskip\hsize minus\hsize\nobreak\hbox{}}

\def\notation{\paragraph*{Notation}}
\def\hypothesis{\paragraph*{Assumption}}


\def \caseparagraph{\@startsection
 {case}{6}{\z@}{1.5ex plus .3ex minus .2ex}{-1em}{\normalsize\bf}}
\def \subcaseparagraph{\@startsection
% {subcase}{7}{\parindent}{1.5ex plus .3ex minus .2ex}{-1em}{\normalsize\bf}}
 {subcase}{7}{\z@}{1.5ex plus .3ex minus .2ex}{-1em}{\normalsize\bf}}


\if \csname datefrench\endcsname\relax
\def \langcase {Case}
\def \langsubcase {Subcase}
\else
\def \langcase {Cas}
\def \langsubcase {Sous-cas}
\fi

\def\case#1{\caseparagraph*{\langcase\space #1:}}
\def\othercase#1{\caseparagraph*{#1:}}
\def\mathcase#1{\caseparagraph*{\langcase\space $#1$:}}
\def\subcase#1{\subcaseparagraph*{\langsubcase\space #1:}}

% Send a warning when running latex
\def\warning#1{\immediate\write0{Warning: #1}}

\newenvironment{care}{\bwarning}{\ewarning}
\def\bcare{\trivlist \item[\hskip \labelsep {\bf Warning!}]}
\def\ecare{\endtrivlist}

\def\subst#1#2{^{#1} \!/\! _{#2}}

\def \comment #1{{\em #1}}
%\def \comment #1{}
\makeatother


\makeatletter
%%% To change size in math mode

\let\Ts=\textstyle
\let\Ds=\displaystyle
\let\Ss=\scriptstyle

%%% Abrevs

\let \comp \circ
\let \wild \_


%%% To put text between two formulas on the same line

\def\text#1{\qquad\hbox{#1}\qquad}

%%%  Macros for mathematics

%% Special symbols

\def \nat {{\mathord{I\kern -.33em N}}}
\def \natb {{\mathord{I\kern -.33em\overline{N}}}}

\def\Vdash {\mathrel{\hbox{\vrule\kern .1ex$\vdash$}}}
\def\VVdash {\mathrel{\hbox{\vrule\kern .2ex\vrule\kern .1ex $\vdash$}}}

% An open box to end demonstrations with

\def\Box{\hbox
  {\vbox
    {\hrule height .033em
     \hbox to .5em
       {\vrule width .033em
        \hbox{\vbox to .5em{\vss}\hss}
        \vrule width .033em}
     \hrule height .033em
    }}}


%% Balanced symbols

\def\absolute#1{\mathopen\mid#1\mathclose\mid}

%% Operators

\def \domain {\dv}
%\def \domain {\dvmathop{\sl dom\,}}
\def \image {\mathop{\sl im\,}}
\def \modulo {\mathbin{\rm modulo}}
\def \inverse{{}^{-1}}
\def \cardinal {{\sl Card\,}}
%\def \restrict #1{\mathbin{\mid}_{#1}}
\def \restrict
   {\mathbin{\hbox{$\mid$\kern-.24em\lower.09em\hbox{\rm\char'22}$\!$}}}
\let \arity = \varrho


%% Quantifiers

\def \fall #1,{\forall #1,\;}
\def \exist #1,{\exists #1,\;}
\def \qt #1.{\forall\,#1.\,}
\def \ex #1.{\exists\,#1.\,}

%% Arrows

\let\ar=\rightarrow
\let\dar=\Rightarrow
\def\imply{\Longrightarrow}


\def \rewrite {\mathrel{-}\joinrel\leadsto}


%% Equalities

%\def \axiom {\mathrel {\mathop =^!}}
\def \axiom {=}
\def \eqdef {\mathrel {\mathop {=\joinrel=}\limits^{def}}}

%% Sizes

\def\({\left(}
\def\){\right)}

\def\Mid{\mathrel{\Big|}}

%% Indices  see also indices.def

\def \ind #1#2{_{#1\in[1,#2]}}

%%

%% names 

\let \th \vdash
\let \tth \Vdash
\let \ttth \VVdash
\let \thh \models

\def\Cal#1{{\cal #1}}

\let \eset = \emptyset



%% Aligned constructions in math mode

% For internal use only

\def\noskip{\lineskiplimit=\the\lineskiplimit
                \lineskip=\the\lineskip}
\def\moreskip{\lineskiplimit=0.4em\lineskip=0.4em plus 0.2em}
\def\moremoreskip{\lineskiplimit=1.2em\lineskip=1.2em plus 0.2em}

\def\vatrix#1{\null\,\vcenter{\normalbaselines\moreskip\m@th
    \let \and \cr
    \ialign{$\displaystyle##$\hfil
      &\quad$\displaystyle##$\hfil&\quad##\hfil\crcr
      \mathstrut\crcr\noalign{\kern-\baselineskip}
      #1\crcr\mathstrut\crcr\noalign{\kern-\baselineskip}}}\,}

% A sequence of expressions is writen on separate lines to form one expression

\def \band #1{\left\{\vatrix{#1}\right.}          % brace and
\def \wand #1{\wedge\band{#1}}                    % wedge  and
\def \vand #1{\vee\band{#1}}                      % vee  and
\def \vor  #1{\vee\left[\vatrix{#1}\right.}       % vee and or
\def \Wand #1{\bigwedge\band{#1}}                 % big wedge and
\def \Vand #1{\bigvee\band{#1}}                   % big vee and
\def \Wor  #1{\bigvee\left[\vatrix{#1}\right.}    % big vee and



% A general usage tabulation in math mode.
%     Indentation is alternatively left and right, so that good place for
%     tabs will produce many different

% \dalign { & blabla :& blabla    \cr 
%           &    bla :& blablabla \cr
%         }
 
% \dalign {  blabla :&& blabla    \cr 
%            bla :   && blablabla \cr
%         }
 
% \dalign { & blabla :&    blabla \cr 
%           &    bla :& blablabla \cr
%         }

% Internal and personal use
\def\Dalign#1#2{\vcenter
 \bgroup
   \let\>=\qquad
   \def\title##1{\crcr\noalign{\medskip}\multispan#1{\hfil\bf##1\hfil}}
%   \halign{$##$\hfil\qquad&&\hfil$##{}$&${}##$\hfil\qquad\cr#2\crcr}
%   \halign{$##$\hfil&&\qquad\hfil$##{}$&${}##$\hfil\cr#2\crcr}
   \halign{&${}##$\hfil&\hfil$##{}$\cr#2\crcr}
 \egroup}
% End internal use

\def\align#1{\vcenter
 \bgroup
   \let\>=\qquad
   \halign{#1\crcr}
 \egroup}

%\def\aligndef#1#2{\expandafter\def\csname aligndef#1\endcsname{#2}}
%\aligndef ]{\argalign}
%\aligndef l{&\hfil$##{}$}
%\aligndef r{&${}##$\hfil}
%\aligndef c{&\hfil$##$\hfil}
%\aligndef -{\qquad}
%\aligndef L{&\hfil##$}
%\aligndef R{&##\hfil}
%\aligndef C{&\hfil##\hfil}
%\def\argalign#1{\relax}
%\def\prealign#1{\csname aligndef#1\endcsname\prealign}
%\def\doalign#1]{\dohalign{\prealign#1]}}
%\def\dohalign#1#2{\expandafter\halign{#1\cr#2\crcr}}



%\def\dalign#1{\ifx#1[\doalign\else\Dalign3{#1}\fi}
\def\dalign{\Dalign3}


\def\Strut{\raise .4em\hbox{\strut}\raise -.4em\hbox{\strut}}

\def \deftitle #1
   {\def \title ##1
      {\crcr\noalign{\medskip}\multispan#1{\hfil\bf##1\hfil}}}



\bgroup
 \catcode `\^^M \active
 \gdef \obeycr {\catcode `\^^M\active \let ^^M\crcr}
\egroup


\ifx \csname linewidth\endcsname \hsize 
 \let  \realwidth \hsize 
\else
 \let \realwidth \linewidth
\fi

\def \oris {\crcr \mid&}
\def \syntaxtable #1
  {\def\is {::=&}
   \def \\{\cr\noalign{\medskip}}\def\cont{\cr&}\vbox 
   \bgroup
   \tabskip=100pt plus 1000pt minus 1000pt
   \halign to \realwidth
       {\hfil$##{}$\tabskip=0pt&
        ${}##$\hfil \tabskip=100pt plus 1000pt minus 1000pt&
        &##\hfil \cr#1\crcr}
   \egroup}


\def \assertions #1
  {\deftitle2 \vbox
   \bgroup
     \halign {\hfil$##{}$&$##$\hfil&&\quad##\hfil\cr#1\crcr}
   \egroup}


%%% old version

\def\malign#1{\vcenter\bgroup
   \deftitle #1
   \def \lign ##1\cr {\noalign {##1}}
   \def \textlign##1{\noalign {\medskip \vbox {\noindent ##1}\medskip}}
   \tabskip=100pt plus 1000pt minus 1000pt
   \halign to \realwidth
       {\hfil$##{}$\tabskip=0pt&
        ${}##$\hfil \tabskip=100pt plus 1000pt minus 1000pt&
        &##\hfil&$##$\hfil&  \cr#1\crcr\egroup}}

\newif \ifhsize \hsizefalse

\newskip \nullskip \nullskip 0pt
\newskip \fullskip \fullskip 100pt plus 1000pt minus 1000pt
\def \maligntrue {\tabskip \fullskip
                   \halign to \realwidth \bgroup}
\def \malignfalse {\halign \bgroup}
\def \malign {\vcenter \bgroup \ifhsize \expandafter \maligntrue \else \expandafter \malignfalse \fi}
\def \endmalign {\crcr \egroup \egroup}

\def \makeword #1#2#3{\expandafter
   \def \noexpand#3{#2 {#1\expandafter \let \expandafter \dummy\string #3}}}

\let \vect \overrightarrow

\def \uad {\hskip 0.5em}


\newskip \mathandskip 
\newskip\mathmargin \mathmargin  0em

\mathandskip 2em plus 0.5fil minus 0.5em
\def \mathand {\hskip \mathandskip}
\def \mathleftright {\hfill{}\hskip 0em plus -2fill\penalty 50{}\hfill}
\def \mathandcr {\penalty 50\mathand}
\def \mathcr {\penalty -10000\mathand}
\def \mathcrleft {\hfill{}\hskip 0em plus -2fill\penalty -10000{}\hskip
     \mathmargin}
\def \mathbreak {\penalty 50}

\def \incrmargin #1{\advance\mathmargin by #1em\advance\mathmargin by #1em}
\def \mathslash#1{\let \do \mathcrleft
   \ifx #1[\let\do\mathdoskip\fi\do#1}
\def \mathdoskip#1#2]{\incrmargin {#2}\mathcrleft}

% Type inference rules

\let \mathmoreskip=\moremoreskip
\def \nomoreskip {\let \mathmoreskip \relax}

\def \bmaths {\let \and \mathandcr  \hsize \realwidth
    \centering \vbox \bgroup \mathmoreskip  \noindent 
    $\displaystyle \let \\ \mathslash}
\def \emaths {\unskip$\egroup}
\def \mathflushleft {\def \emaths {\hfill$\egroup}\nomoreskip}
\newenvironment {maths}{\bmaths}{\emaths}

\def \maths #1{\bmaths #1\emaths}

\def \whereis {\leftarrow}
\def \where #1{[{\let \is \whereis #1}]}
\def \dowhere #1\is#2\is{[\ifx#2\empty #1\else #2/#1\fi]}
\def \where #1{\let \is \relax\dowhere #1\is\empty\is}

\let \closeplus\relax
\let \Empty \relax
\let \is \relax
\def \plus #1{\PLUS #1\is\is\END}
\def \PLUS #1\is#2\is#3\END{\oplus\ifx #3\is (#1\plustoken#2)\else #1\fi}
\let \plustoken \mapsto

\let \diverge \uparrow

\def \set #1{\{#1\}}
\def \sem #1{\lbrack\!\lbrack#1\rbrack\!\rbrack}


\def\mathrm #1{\mathrel {\rm #1}}
\let \coerce \leq
\makeatother

\makeatletter


%% see infer.doc for a documentation

% Judgements

\let\th=\vdash

% naming

\expandafter \ifx \csname name\endcsname\relax
   \def\name#1{\hbox{\sc #1}}\else \relax \fi

\def \labname #1{{\rm(\name{#1})}}
\let \lab = \labname
\def \clab #1{\hbox {$\smash {\lab{#1}}$}}
\newskip \beforelabskip \beforelabskip = 0.5em
\def \eqlab {\eqno \lab}

%% inside rules

\def \nmir #1[#2]{\crcr \hrulefill \cr}
\def \rmir #1[#2]{\crcr \hrulefill & \clab {#2} \cr}
\def \lmir #1[#2]{\relax}                         
\def \Lmir #1[#2]{\crcr \Llabo {#2}\hrulefill \cr}     
\def \Rmir #1[#2]{\crcr \hrulefill \Rlabo {#2}\cr}    
\def \imir {\crcr \hrulefill\cr}

% outside rules

\def \rlabo #1{\hskip \beforelabskip \clab {#1}} 
\def \llabo #1{\clab {#1}\hskip \beforelabskip}
\def \Rlabo #1{\rlap {\rlabo {#1}}}   
\def \Llabo #1{\llap {\llabo {#1}}}   
\let \labr \rlabo \let \labl \llabo \let \labR \Rlabo \let \labL \Llabo


\def \rlab #1 {\rlap {\lab {#1}}}
\def \llab #1 {\llap {\lab {#1}}}


% A good separator for multiple hypothesis

\let \and \qquad
\let \andcr \cr


% Single rules

\def \lessskip  {\lineskiplimit=.3em\lineskip= .2em plus .1em}

\makeatletter
\def \mathrulename {\scriptsize \labname}
\def \ignoredash #1{\let\next\ignoredash
         \ifx #1-\relax\else\let\next#1\fi\next}
\def \mathrule {\@ifnextchar [{\mathrulelab}{\@mathrule}}
\def \mathrulelab[#1]#2{\hbox
   {\vbox {\hbox {\mathrulename {#1}}\nointerlineskip\smallskip
           \hbox {$\@mathrule{#2}$}}}}
\def \@mathrule #1{{\let \and \qquad \let \rlab \rl \let \llab \ll
     \def \mir {MiR}\def\endrule {EnDrUlE}
     \displaystyle \offinterlineskip \lessskip  \bgroup \openrule 
     \beginrule #1\mir \mir \endrule}}
\def \beginrule #1\mir#2{#1\ifx #2\mir \let \next \finishrule 
     \else \def \next {\middlerule\openrule \beginrule\ignoredash#2}\fi\next} 
\def\finishrule#1\endrule{\closerule\egroup}

\def \openrule {\hbox \bgroup \vbox \bgroup
     \halign \bgroup \hfil $##$\hfil &\hskip \beforelabskip$##$\hfil \cr}
\def \closerule {\crcr\egroup\egroup\egroup}
%\def \endrule {\closerule \egroup}

\def \singleRule {\crcr\noalign {\vskip -0.1em}\closerule \over}
\let \middlerule \singleRule


% Double and Tripple rules see infer-more.def

% Logic and rewriterules

\def \logicrule #1{\let \mir \Longrightarrow #1}


\def \longleadsto {\mathrel{-}\joinrel\leadsto}

\def \rewriterule #1{\displaystyle \offinterlineskip \lessskip
     \bgroup\let \mir \longleadsto \openrule#1\endrule}

\def \suspend #1{\vbox to #1 {\leaders \hbox {$\cdot$}\vfill}}


\if \relax \csname keywordstyle\endcsname  \let \keywordstyle \tt \fi

%\def \keyword #1{{\hbox {\keywordstyle #1}}}
\def \keyword #1{{\keywordstyle #1}}
\def \mathtoken #1{\ifmmode
   \mathopen{}\keyword {#1}\mathclose{}\else \keyword{#1}\fi}
\def \mathpostfix #1{\mathrel{\keyword {#1}}\mathclose{}}
\def \mathprefix #1{\mathopen{}\mathrel{\keyword {#1}}}
\def \mathinfix #1{\mathrel{\keyword {#1}}}

\def\ignorefirst #1{}
\def \lwt #1{\expandafter \lowercase \expandafter \bgroup 
  \expandafter \ignorefirst \string #1\egroup}


\def \makeprefix #1#2{\expandafter\def \noexpand #1{\mathprefix  {#2}}}
\def \makepostfix #1#2{\expandafter\def \noexpand #1{\mathpostfix  {#2}}}
\def \maketoken #1#2{\expandafter\def \noexpand #1{\mathtoken {#2}}}
\def \makeinfix #1#2{\expandafter\def \noexpand #1{\mathinfix {#2}}}

\def \makein #1{\expandafter\def
  \noexpand #1##1in{\keyword {\lwt #1}\,\,##1\,\,\keyword {in}\,\,}}

\let \@ \;
\def \Let #1=#2in{\keyword {let}\@ {#1 = #2} \@\keyword {in}\@}
\def \fun (#1){\keyword {fun}\@ #1 \rightarrow}
\def \type #1=#2in{\keyword {type}\@ #1 = #2 \@\keyword {in}\@}

\makeatother

\let \cont \kappa
\let \t \tau
\let \l \ell
\let \s \sigma
\let \tv \alpha
\let \vect \bar

\def \Gen {{\rm Gen}}

\def \tplus #1{\TPLUS #1\is\is\END}
\def \TPLUS #1\is#2\is#3\END{\oplus\ifx #3\is (#1:#2)\else #1\fi}

%%%%%%%%%%%%%%% light programs

\makeatletter
\def \progstyle {\tt}
\newskip \programindent \programindent 4em
\def \program {\par \begingroup 
    \medskip \parindent \programindent \@tempswafalse
    \frenchspacing \@vobeyspaces \let \par \programpar
    \obeylines \progstyle
    \catcode``=13 \@noligs \let \do \@makeother \dospecials 
    \catcode `\\ 0 \catcode `\{ 1 \catcode `\} 2
    \prog@specials \prog@initspecials \obeyspaces}

\def \programpar {\if@tempswa\hbox{}\fi\@tempswatrue\@@par}

\bgroup
\gdef \less@specials {\relax
 \catcode `| \active 
 \catcode `< \active
 \catcode `> \active
 \catcode `- \active
 \catcode `\' \active 
}
\gdef \more@specials {\relax
 \catcode `( \active
 \catcode `) \active
}
\gdef \prog@specials {\less@specials\more@specials}
\prog@specials
\gdef \prog@initspecials {\def
    -{\minus@program}\def
    '{\acurate@program}\def
    |{\bar@program}\def
    <{\less@program}\def
    >{\greater@program}\def
    ({\less@program}\def
    ){\greater@program}}
 \gdef \bar@@program {\let\do \eattoken
  \ifx \token >$\droite$\else ${}\mid{}$\let \do \relax\fi \do}
 \gdef \minus@@program {\let\do \eattoken
  \ifx \token >$\ar$\else -\let \do \relax\fi \do}
\egroup

\def \acurate@program {\futurelet \token \acurate@@program}
\def \acurate@@program {\let \do \eattoken
  \ifx \token a$\alpha$\else 
   \ifx \token b$\beta$\else 
    \char `\' \let \do \relax\fi\fi \do}
\def \less@program {\hbox{$<$}}
\def \greater@program {\hbox{$>$}}
\def \bar@program {\futurelet \token \bar@@program}
\def \minus@program {\futurelet \token \minus@@program}


\def \endprogram {\ifdim\lastskip >\z@ \@tempskipa\lastskip 
       \vskip -\lastskip\fi\medskip \endgroup
     \@endpetrue}

\def\prog {\begingroup \hbox \bgroup
  \catcode ``=13 \@noligs
  \let \do \@makeother \dospecials \progstyle
   \prog@specials \prog@initspecials
    \catcode `\\ 0 \catcode `\{ 1 \catcode `\} 2
  \@prog}
\def\@prog#1{\obeyspaces \frenchspacing \catcode `\  10
            \def\@tempa ##1#1{##1\egroup\endgroup}\@tempa}
\makeatother

%%%%%%%%%%%%%%%%%%%  "quotations"

\makeatletter
% on  active  le cararectere " et on s'en sert pour le mode \prog
% s'il est d�fini ou \verb autrement

\bgroup
\@makeother \"
\gdef \progquote {\prog"}
\catcode `\" \active
\gdef "{\progquote}
\egroup

\def \quoteon {
\let \quote@sanitize \@sanitize
\def \@sanitize {\quote@sanitize \@makeother\"}
\let \quote@specials \dospecials 
\def \dospecials {\quote@specials \do\"}
\def \quoteon {\catcode `\" \active}\quoteon}

\def \quoteoff {\@makeother \"}
\makeatother



%%%%%%%%%%%%%%%%
\def \lift #1{\;{}^{#1}}
\def \inu #1{\lift {i \in 1..#1}}
\def \jnu #1{\lift {j \in 1..#1}}

%% typo

\mathchardef \le "313C
\mathchardef \ge "313E

\mathcode `< "4268
\mathcode `> "5269


\catcode `\| \active 
\let \olspecials \dospecials \def \dospecials {\olspecials \do \|}

%\def \droite {\mathrel {\,\hbox {\large$\triangleright$}\,}}
\def \droite {\mathrel{\tt =}}

\makeatletter
\def |{\baractive}
\def \baractive {\futurelet\token \next@baractive}
\def \eattoken #1{}
\def \next@baractive{\let\do\eattoken
    \ifx \token>\droite\else \mid \let \do \relax \fi \do}


\def \Def #1in{\mathprefix {let}#1\mathinfix{in}}
\def \Let#1in{\mathprefix {let}#1\mathinfix{in}}
\def \Reply#1to{\mathprefix {reply}#1\mathinfix{to}}
\def \>{\droite}
\def \And {\mathrel{\tt and}}
\def \tuple #1{{\tt (}#1{\tt )}}
\def \<#1> {\tuple {#1}}

\let \bcup \cup
\let \thd \th
\def \fv {fv}
\def \defines {::}
\def \produces #1{\Ds\mathop{\Longrightarrow}^{#1}}
\def \reduces {\longrightarrow}
\let \red \reduces
\let \redcham \mapsto

\let \plus \tplus

\makeatother
 

\def \dd {{\cal D}}
\def \pp {{\cal P}}

\def \dv {dv}

\def \redsoupe {\mathrel {\mathop {\red} \limits^s}}
\let \redstruct \rightleftharpoons
\let \redchaud \rightharpoonup
\let \redfroid \leftharpoondown
\let \cool \leftharpoondown
\let \heat \rightharpoonup

\def\redstep{\mathrel{\mathpalette\redstepaux{}}}
\def\redstepaux#1{\vcenter{\hbox{\ooalign{\hfil
      \raise2pt \hbox{$\rightharpoonup$}\hfil\cr\hfil
      \lower2pt\hbox {$\leftharpoondown$}\hfil\crcr
      $\longrightarrow$}}}}


\let \redsoupe \red
\let \redx \red


\def \Comment #1{}
\let \agree \approx
\quoteon

\maketoken \int {int}





\def\showgraph{\par\medskip\centerline{\begin{toimage}
\box\graph
\end{toimage}
\imageflush}\medskip}
\def\status#1{{\tt #1}}

\title{Compiling Join-Patterns
\footnote{
This work is partly supported by the ESPRIT 
    CONFER-2 WG-21836}}
\author {Luc Maranget \and Fabrice
Le~Fessant\\[.5em]
INRIA Rocquencourt, 
    BP 105, 78153 Le Chesnay Cedex France.
\\[.5em]
{\tt\small \{Luc.Maranget, Fabrice.Le-Fessant\}@inria.fr}
}

\date {}


\htmlfoot{\rule{}{}\begin{center}DVI version \ahrefurl{pat.dvi}\end{center}}
\htmlhead{\begin{center}DVI version \ahrefurl{pat.dvi}\end{center}\rule{}{}}
\toplinks{suite.html}{index.html}{env.html}
\begin{document}



\maketitle
\thispagestyle{empty}

\begin{abstract}
  
  The join-calculus is both a name passing calculus and a core
  language for concurrent and distributed programming.  An essential
  part of its implementation is the compilation of {\em
    join-patterns}.  Join-patterns define new channels and all the
  synchronizations they take part to at the same time.  Relying on the
  experience based on our two implementations, we study the
  translation of join-patterns into deterministic finite-state
  automata as well as some related optimizations.

  
\end{abstract}
\tableofcontents

\section {Introduction}

%%Baratin
Join-pattern is the distinctive feature of the join-calculus, seen
both as a process calculus and as a programming language.  On the
calculus side, join-calculus can roughly be seen as a functional
calculus plus join-patterns, thus achieving the same expressive power
as previous name-passing process calculi~\cite{Milner-pi-calculus}.
Join-definitions are made of several clauses, each clause being a pair
of a join-pattern and of a guarded process.  A join-pattern expresses
a synchronization between several {\em names} (or channels). When
messages are pending on all the names that appear in a given
join-pattern, then the corresponding clause is said to be active and
its guarded process {\em may} be fired.  A definition whose
join-patterns share some names expresses sophisticated
synchronizations. In such a definition, a message on a name that
appears in several active clauses is consumed as soon as one of the
corresponding guarded processes is fired.

Join-languages are built on top of the join-calculus taken as a core
language.  Therefore, names are first-class citizens, computations are
first abstracted as collections of asynchronous processes, and
join-patterns provide an unique, clear and powerful mechanism for
synchronizing these computations.  The documentation for the
join-language~\cite{Join} includes a tutorial that shows how join
definitions may encode classical synchronization constructs such as
locks, barriers, shared counters,\ldots

%%%Conclusion de l'intro
On the implementation side, join-patterns are meant to be heavily used
by programmers, as the only synchronization primitive available.
Thus, their compilation requires much care.  At the moment, we propose
two compilers: the {\tt join} compiler~\cite{Join}, a language of its
own, and the {\tt jocaml} compiler~\cite{Jocaml}, an extension of the
Objective Caml functional language.

%%% Plan
Section~\ref{join-calculus} of this paper succinctly presents the
join-calculus syntax and semantics. Then, section~\ref{automatas}
introduces the kind of automata we use to compile
join-synchronization, while section~\ref{implementation} presents two
techniques for implementing them.  The first technique directly
derives from automata description and is used in our {\tt join}
compiler. The second technique performs some extra runtime tests, this
is the technique used in our {\tt jocaml} compiler.
Sections~\ref{optimization} and~\ref{fusion} discuss
optimizations and section~\ref{conclusion} concludes.




\section{A rapid tour of the join-calculus}
\label{join-calculus}
\subsection {Syntax}

We here rephrase the traditional presentation of the core
join-calculus~\cite{FournetGonthier96}, where names are
the only value. Thus, we ignore the issue of system primitives and
constants, since
names provide sufficient expressive power for our purpose of
describing our implementation of pattern matching (However we use
primitives and constants in our examples).
We slightly change the syntax of~\cite{FournetGonthier96}, in order to
match the one of
the {\tt join} programming language.

We use $x$ to denote a name in general.
$$
\begin{array}{l@{\hspace{2cm}}l}
\begin{array}{rcl}
P & ::= & x\tuple{x_i\inu p}\\
  & \mid & \Def D in P\\
  & \mid & P | P
\end{array}
&
\begin{array}{rcl}
D & ::= & J |> P \\
  & \mid & D \And D\\[1em]

J & ::= &  x\tuple {x_i\inu p}\\
  & \mid & J | J
\end{array}
\end{array}
$$

A process $P$ is either a message, a defining process, or a parallel
composition of processes (note that names are polyadic, meaning that
messages may be made of several values);
%
a definition $D$ consists of one or several clauses $J = P$ that
associate a guarded process $P$ to a specific message pattern $J$;
%
a join-pattern $J$ consists of one or several messages in parallel. We
say that the pattern $J = \ldots x\tuple {x_i \inu p}\ldots$
defines the name $x$ and that a definition defines the set of the names
defined by its patterns.
Moreover, patterns are linear, i.e. names may appear at most once in
a given pattern.

Processes and definitions are known modulo renaming of bound
variables, as substitution performs $\alpha$-conversion to avoid
captures.

\subsection{Semantics}


This semantics is specified as a reflexive chemical abstract machine
(RCHAM)~\cite{BeBo92}. The state of the computation
is a {\em chemical soup} $\dd \tth \pp$ that consists of two
multisets: active definitions~$\dd$ and running processes~$\pp$.

The chemical soup evolves according to two families of rules: {\em
  Structural rules} $\redstruct$ are reversible ($\heat$ is heating,
$\cool$ is cooling); they represent the syntactical rearrangement of
terms (heating breaks terms into smaller ones, cooling builds larger
terms from their components).  {\em Reduction rules} $\redsoupe$
consume specific processes present in the soup, replacing them by some
others; they are the basic computation steps.
%

We present simplified chemical rules
(see~\cite{FournetGonthier96,FGLMR96_Calculus_mobile_agent} for
the complete set of rules).
Following the chemical tradition, every rule applies on any matching
subpart of the soup, non-matching sub-parts of the soup being left
implicit.
$$
\let \N \name
\let \su \varphi
\begin{array}{rl@{~}c@{~}rl@{\quad}l}
  &\tth P_1 | P_2   &\redstruct&           &\tth P_1, P_2 &\mbox{S-Par}\\
D_1 \And D_2 &\tth             &\redstruct& D_ 1, D_2 &\tth   &\mbox{S-And}\\
             &\tth \Def D in P &\redstruct&         D  &\tth P
             &\mbox{S-Def} \\[1em]
      J |> P &\tth \su  (J)    &\redsoupe & J |> P    &\tth \su (P)
      &\mbox{R-$\beta$}
\end{array}
$$
Two of the rules above have side conditions:
\begin{itemize}
\item (\name{S-Def}) 
the names defined in $D$ must not appear anywhere
in solution but in the reacting process and definition~$D$ and $P$.
This condition is global; in combination with $\alpha$-renaming it
enforces lexical scoping.

\item (\name{R-$\beta$}) $\varphi (\cdot)$ substitute actual names for
the received variables in $J$ and $P$.

\end{itemize}
Additionally, we only consider well-typed terms and reductions.
See~\cite{join-typing-97} for details on a rich polymorphic type system
for the join calculus. Here, this mostly amounts to assuming that
message and name arity always agree.


In this paper, we take particular interest in the reduction (\name{R-$\beta$}).
Informally, when there are messages pending on all the names defined
in a given join-pattern, then the process guarded by this join-pattern
may be fired. When firing is performed, we say that a {\em matching} occurs.
On the semantics level, there is a message $\tuple{x_i\inu p}$ pending
on a name~$x$ when there is an active molecule
~$x\tuple {x_i\inu p}$ in the chemical soup.

Thus, we may define the reactivity status of a given chemical soup as
the multiset of the active molecules in it.
Later on in this paper, we shall consider various abstractions of this
reactivity status.


\subsection{The join programming languages}

Apart from primitives, join-languages support {\em synchronous}
names, which the core join-calculus does not provide directly.
Synchronous names send back results, a bit like functions.
However synchronous names may engage in any kind of
join-synchronization, just as asynchronous names do.
A program written using synchronous names can be translated
into the core join-calculus alone. 
The translation is analogous to continuation passing style
transformation in the $\lambda$-calculus.
In our implementation, as far as pattern matching is concerned,
a synchronous name
behave like it was asynchronous and carried one additional continuation
argument. All implementation difficulties concentrate in managing this extra
argument, whose presence had no effect on pattern matching itself.



The {\tt join} language~\cite{Join} is our first prototype. All
examples in this paper are in {\tt join} syntax.  The system consists
in a bytecode compiler and a bytecode interpreter.  Both compiler and
interpreter are Objective Caml~\cite{Objective-Caml} programs and it
is easy to lift Objective Caml data types and functions into {\tt
  join} abstract data types and primitives.  For instance, {\tt join}
programs easily draw graphics, using the graphics Objective Caml
library.  As a consequence, {\tt join} can be seen either as a
language of its own, featuring many primitives, or as a distributed
layer on top of Objective Caml.  Continuations are encoded using
ad hoc threads, which are created and scheduled by the {\tt join}
bytecode interpreter.


The {\tt jocaml} system is our second implementation. In {\tt jocaml},
all join-calculus constructs for concurrency, communication,
synchronization and process mobility are directly available as
syntactical extensions to Objective Caml.  On the runtime environment
side, we have supplemented the original Objective Caml runtime system
(which already provides a thread library) with a special ``join''
library and a distributed garbage collector~\cite{GCDistri}. On the
compiler side, the Objective Caml compiler has been extended to
translate join-calculus source code into functions calls to the
``join'' library.  However, we also introduced a few new instructions
to Objective Caml bytecode, but only to handle code mobility, a
feature orthogonal to pattern matching.  The {\tt jocaml} system is
currently available as a prototype version~\cite{Jocaml}.

\section{Pattern matching in join definitions}
\label{automatas}
\subsection{Principle}

Consider the following join definition:
\begin{verbatim}
let A(n) | B() = P(n)
and A(n) | C() = Q(n)
;;
\end{verbatim}
This defines three names $A$, $B$ and $C$. Name $A$ has arity one,
whereas names $B$ and $C$ have arity zero.  Names may be synchronous
or asynchronous, depending on whether there are \verb+reply ... to
...+ constructs applying to them inside the guarded processes~$P(n)$
and $Q(n)$ or not.

According to the general join-calculus semantics, the guarded process
$P(n)$ may be fired whenever there are some messages pending on $A$ and
$B$. Similarly, $Q(n)$ may be fired whenever there are some messages
pending on $A$ and $C$. In both cases, the formal parameter $n$ is
replaced by (or bound to in the implementation) one of the messages
pending on $A$.

Reactivity information is to be considered at the definition level,
since matching is indeed performed at this level.  Moreover, in order
to use finite-state automata, we want this information to range on a
finite set of possible values.  As far as matching is concerned and by
the linearity of patterns, only the presence or absence of messages
matters.  Thus, let us call \status{0} the status of a name without
any message pending, and \status{N} the status of a name with at least
one message pending.  Then, the status of a definition is a tuple
consisting of the statuses of the names it defines, once some
arbitrary order of these names has been adopted.

For instance, if some messages are pending on $B$ and $C$, whereas
none is pending on $A$, then the status of the $A$, $B$, $C$
definition is a three-tuple written \status{0NN}.

A {\em matching status} is defined as a status that holds enough
\status{N}, so that at least one guarded process can be fired.

Definition status evolves towards matching status as messages arrive.
This yields a first kind of ``increasing'' transitions.  More
specifically, when a message arrives on some name, then this name
status either evolves from \status{0} to \status{N} or remains
\status{N}.  Definition status evolves accordingly.  In the $A$, $B$,
$C$ case we get the following transitions.  (In this diagram,
transitions are labeled by the name that gets a new message and
matching statuses are filled in gray.)
\begin{toimage}
.PS
cm = 0.3937;
pt = 1/72;
hu = 2 * cm;
vu = 1.5 * cm
circlerad = 12 * pt
arcrad = circlerad
rblabht = 2 * circlerad
rblabwid = 4 * circlerad
fillval=0.3
pi=3.1415927

define clab {
    {circle at Here $1 $2 $3}
}

define ilab {
    {box invis $1 $2 $3 at Here}
}

define ln {
    L: [circle rad 0]
    if $4 < 0 then {
      box $3 rjust wid 0 ht 0 at L.c + ($1*(hu/3)-circlerad/4, $2*(vu/3))
    } else {if $4 > 0 then {
      box $3 ljust wid 0 ht 0 at L.c + ($1*(hu/3)+circlerad/4, $2*(vu/3))
    } else {
      box $3 ljust wid 0 ht 0 at L.c + ($1*(hu/3)+ 0.001 * cm, $2*(vu/2.5))
    }}
    {line  ->  chop from L.c to L.c + ($1*hu, $2*vu)}
    move to  L.c + ($1*hu, $2*vu)
}

define dln {
    {line  -> dotted right $1*hu up $2*vu chop}
    move right $1*hu up $2*vu
}

define htccw {
  psi=$1
  phi=$2
  {
  c0=circlerad * 2 * cos(psi/2) * cos(phi+psi/2)
  s0=circlerad * 2 * cos(psi/2) * sin(phi+psi/2)
  c1=circlerad * cos(phi) ; s1 = circlerad * sin(phi)
  c2=circlerad * (2 * cos(psi/2) + 1) * cos(phi+psi/2)
  s2=circlerad * (2 * cos(psi/2) + 1) * sin(phi+psi/2)
  c3=circlerad * cos(phi+psi) ; s3 = circlerad * sin(phi+psi)
  A: [circle rad 0.0]
  move to A.c + (c1,s1)
  arc $4 rad circlerad to A.c + (c2,s2)
  arc -> $4 rad circlerad to A.c + (c3,s3)
  circle invis $3 with.c at A.c + (c0,s0)}
}

define htcw {
  psi=$1
  phi=$2
  {
  c0=circlerad * 2 * cos(psi/2) * cos(phi-psi/2)
  s0=circlerad * 2 * cos(psi/2) * sin(phi-psi/2)
  c1=circlerad * cos(phi) ; s1 = circlerad * sin(phi)
  c2=circlerad * (2 * cos(psi/2) + 1) * cos(phi-psi/2)
  s2=circlerad * (2 * cos(psi/2) + 1) * sin(phi-psi/2)
  c3=circlerad * cos(phi-psi) ; s3 = circlerad * sin(phi-psi)
  A: [circle rad 0.0]
  move to A.c + (c1,s1)
  arc cw $4 rad circlerad to A.c + (c2,s2)
  arc cw -> $4 rad circlerad to A.c + (c3,s3)
  circle invis $3 with.c at A.c + (c0,s0)}
}

define tourccw {
  htccw(pi/2,$1,$2,$3)
} 

define tourcw {
  htcw(pi/2,$1,$2,$3)
}

hu=3*cm
[
vu=2*cm
circlerad=15*pt
clab("\status{000}")
{ln(-1,-1,"$B$",-1) ; clab("\status{0N0}") ; tourccw(pi/2,"$B$")
  {ln(0,-1,"$A$",0) ; clab("\status{NN0}",fill)
   tourccw(pi,"$A, B$")
   ln(1,-1,"$C$",-11)}
  {ln(1,-1,"$C$",1) ; clab("\status{0NN}")
    tourcw(pi/8,"$\,B, C$")
    ln(0,-1,"$A$",-1) ; clab("\status{NNN}",fill)
    tourccw(5*pi/4,"$A,B,C$")}}
{ln(0,-1,"$A$",0) ; clab("\status{N00}") ; tourcw(pi/2.5,"$A$")
  {ln(-1,-1,"$B$",-1)}
  {ln(1,-1,"$C$",1)}}
{ln(1,-1,"$C$",1) ; clab("\status{00N}") ; tourcw(pi/2,"$C$")
  {ln(-1,-1,"$B$",1)}
  {ln(0,-1,"$A$",0) ; clab("\status{N0N}",fill) tourcw(0,"$A, C$")
   ln(-1,-1,"$B$",1)}}
]
.PE
\end{toimage}%$
\showgraph


Definition status also evolves when matching occurs. This yields
new, ``decreasing'', transitions that we call matching transitions.
According to the join-calculs semantics,
matching may occur at any moment (provided of course that
matching is possible).
As a consequence, matching transitions start from matching states and
they are unlabelled.
In the $A$, $B$, $C$ case, they are as follows:
\begin{toimage}
.PS
lab=0.5

define iln {
    move right $1*hu up $2*vu
}

aphi=pi/4

define alncw {
 c= circlerad*cos($1) ; s = circlerad*sin($1)
 c2 =  circlerad*cos($2) ; s2 = circlerad*sin($2)
 C: [circle rad 0]
 if $4 < 0 then {abs = -$4} else {abs = $4}
 x1 = c ; y1 = s ; x2 = $3*hu+c2 ; y2 = $4*vu+s2
 d = (x2-x1) * (x2-x1) + (y2-y1)*(y2-y1)
 ar = sqrt(d) / (2*sin(aphi))
 D: arc cw -> $6 rad ar from C.c + (x1,y1) to C.c + (x2,y2)
 r=last arc.rad
 x3 = C.c.x + lab*(x1+x2) ; y3 = C.c.y + lab*(y2+y1)
 d = sqrt((x3-D.c.x) * (x3-D.c.x) + (y3 -D.c.y)*(y3-D.c.y))
 line invis from D.c to r/d  <D.c,(x3,y3)> ; $5
}

define aln {
 c= circlerad*cos($1) ; s = circlerad*sin($1)
 c2 =  circlerad*cos($2) ; s2 = circlerad*sin($2)
 C: [circle rad 0]
 if $4 < 0 then {abs = -$4} else {abs = $4}
 x1=c ; y1 = s ; x2 = $3*hu+c2 ; y2 = $4*vu+s2
 d = (x2-x1) * (x2-x1) + (y2-y1)*(y2-y1)
  ar = sqrt(d) / (2*sin(aphi))
 D: arc -> $6 rad ar from C.c + (x1,y1) to C.c + (x2,y2)
 r=last arc.rad
 x3 = C.c.x + lab*(x1+x2) ; y3 = C.c.y + lab*(y2+y1)
 d = sqrt((x3-D.c.x) * (x3-D.c.x) + (y3 -D.c.y)*(y3-D.c.y))
 line invis from D.c to r/d  <D.c,(x3,y3)> ; $5
}

clab("\status{000}")
{iln(-1,-1,invis) ; clab("\status{0N0}")
  {iln(0,-1,"$B$") ; clab("\status{NN0}",fill) ;
    {dln(0,1)} ; {dln(1,1)} ; {dln(1,2)}
    tourccw(pi,"",dotted) ; iln(1,-1,"$C$")}
  {iln(1,-1,"$C$") ; clab("\status{0NN}")
    iln(0,-1,"$B$") ; clab("\status{NNN}",fill)
     tourccw(5*pi/4,"",dotted)
     {dln(0,1)} ; {dln(-1,1)} ; {dln(1,1)} ;
     {aln(3*pi/8,-pi/3,0,2,"",dotted)}
     {dln(-1,2)} ; {dln(1,2)}}}
{iln(0,-1,"$B$") ; clab("\status{N00}")
  {iln(-1,-1,"$A$")}
  {iln(1,-1,"$C$")}}
{iln(1,-1,"$C$") ; clab("\status{00N}")
  {iln(-1,-1,"$A$")}
  {iln(0,-1,"$B$") ; clab("\status{N0N}",fill) ;
    {dln(0,1)} ; {dln(-1,1)} ; {dln(-1,2)}
    tourcw(0,"",dotted) ; iln(-1,-1,"$A$")}}
.PE
\end{toimage}
\showgraph


Observe that there may be several matching transitions starting from a
given status.  This is not always a consequence of the
non-deterministic semantics of the join-calculus.

Often, ambiguity is only apparent.  For instance, matching transitions
starting from \status{NN0} lead to \status{NN0}, \status{N00},
\status{0N0} and \status{000}.  When such a matching occurs, two
messages are consumed (one pending on $A$ and one pending on $B$)
then, depending on whether there are some messages left pending on $A$
and $B$ or not, status evolves to \status{NN0}, \status{N00},
\status{0N0} or \status{000}.  From the implementation point of view,
this means that a little runtime testing is required once matching has
been performed. Here, we pay a price for using finite-state automata.

However, some true non-determinism is still present.  Consider
status~\status{NNN} for instance.  Then, both guarded processes of the
$A$, $B$, $C$ definition can now be fired.  The choice of firing
either $P(n)$ or $Q(n)$ will result in either consuming one message
pending on $A$ and one on $B$, or consuming one message pending on $A$
and one on $C$.

Finally, a view of join-matching compilation can be given by taking
together both kinds of transitions. This yields a non-deterministic
automaton.

Note that matching of non-linear patterns  can also be compiled using automata.
For instance, if a name appears at most twice in one or more pattern,
then it status will ramge on $\status{0}$, $\status{1}$ and $\status{N}$.
We do not present this extension in greater detail, for simplicity, and
because we do not implement non-linear patterns.

\subsection{Towards deterministic automata}

For efficiency and simplicity reasons we choose to implement matching
using deterministic automata that react to message reception.

Fortunately, it is quite possible to do so. It suffices to perform
matching as soon as possible.  More precisely, when a message arrives
and carries definition status into matching status, matching is
performed immediately, while definition status is adjusted to reflect
message consumption.  This results in pruning the status space just
below matching statuses.


In practise, in the $A$, $B$, $C$ case, we get the
automaton of figure~\ref{abc}.
\begin{figure}[th]
\begin{toimage}
.PS
circlerad=16*pt
hu=3 *cm
vu=3 *cm
clab("\status{000}")
{ln(-1,-1,"$B$",-1) ; clab("\status{0N0}")
  tourccw(-9*pi/10,"$B$")
  tourccw(pi/2,"$A$",dotted)
  {alncw(pi/4,-3*pi/4,1,1,"$A$",dotted)}
  {ln(1,-1,"$C$",1) ; clab("\status{0NN}")
    htccw(pi/4,-pi/2-pi/8,"$B, C$")
    {alncw(3*pi/4,-pi/4,-1,1,"$A$",dashed)}
    {aln(pi/4,-3*pi/4,1,1,"$A$",dotted)}
    tourccw(-5*pi/12,"$A$",dotted)
    tourcw(-7*pi/12,"$A$",dashed)
}}
{ln(0,-1,"$A$",0) ; clab("\status{N00}")
  tourcw(5*pi/12,"$A$")
  tourccw(-3*pi/4,"$B$",dotted)
  tourccw(-17*pi/12,"$C$",dashed)
  {alncw(pi/2,-pi/2,0,1,"$B$",dotted)}
  {aln(pi/2,-pi/2,0,1,"$C$",dashed)}
}
{ln(1,-1,"$C$",1) ; clab("\status{00N}") ;
  tourcw(-pi/10,"$C$")
  tourcw(pi/2,"$A$",dashed)
  {aln(3*pi/4,-pi/4,-1,1,"$A$",dashed)}
  {ln(-1,-1,"$B$",-1)}}
.PE
\end{toimage}
\showgraph
\caption{\label{abc} Automaton in the $A$, $B$, $C$ case}
\end{figure}

Observe that all transitions are now labeled and that a name labels a
transition when message reception on this name triggers that
transition.  Furthermore, matching transitions that correspond to
firing $P(n)$ or firing $Q(n)$ are now represented differently (the
former by a dotted arrow, the latter by a dashed arrow).  This
highlights the difference between false and true non-deterministic
transitions: real non-determinism is present when there are both
dotted and dashed edges with the same label starting from the same
node.

For instance, there are two $B$-labeled dotted transitions starting
from $\status{N00}$.  Non-determinism is only apparent here, since
$P(n)$ is fired in both cases and that the selected transition depends
only on whether there is at least one message left pending on $A$
or not after firing $P(n)$.


By contrast, from status \status{0NN}, the automaton may react to the
arrival of a message on $A$ in two truly different manners, by firing
either $P(n)$ or $Q(n)$.  This is clearly shown in figure~\ref{abc} by
the  $A$-labeled edges that start from status~\status{0NN}, some of
them being dashed and the others being dotted.  A simple way to avoid
such a non-deterministic choice at run-time is to perform it at
compile-time.  That is, here, we suppress either dotted or dashed
$A$-labeled transitions that start from $\status{0NN}$.

In the rest of the paper, we take automata such as the one of
figure~\ref{abc} as suitable abstractions
of join-pattern compilation output.

\subsection{Automata and semantics}

Both the ``match as soon as possible'' behavior and the deletion of
some matching transitions have a price in terms of semantics.
More precisely, some CHAM behaviors now just cannot be observed
anymore.
However,
the CHAM semantics is a non-deterministic one: an initial
configuration of the CHAM {\em may} evolve into a variety of configurations.
Furthermore, there is no fairness constraint of any kind and
no particular event is required to occur.

As an example of the consequence of the ``match as soon as possible''
behavior, consider this definition:
\begin{verbatim}
let A() = print(1);
and B() = print(2);
and A() | B() = print(3);
;;
\end{verbatim}
Then, we get the following automaton:
\begin{toimage}
.PS
clab("\status{00}")
 tourcw(-3*pi/4,"$A$",dotted)
 tourccw(-pi/4,"$B$",dashed)
.PE
\end{toimage}
\showgraph
Status \status{NN} that is preceded by the two matching statuses
\status{0N} and \status{N0} cannot be reached. As a consequence, the
above program will never print a {\tt 3}, no matter how many messages
are sent on $A$ and $B$.

Next, to illustrate the effect of deleting ambiguous matching
transitions, consider the following definition:
\begin{verbatim}
let A() = print(1);
and A() | B() = print(2);
\end{verbatim}
Such a definition will get compiled into one of the following
deterministic automata:
\begin{toimage}
.PS
{clab("\status{00}")
 tourccw(pi/4,"$A$",dotted)
 {ln(0,-1,"$B$",0)
   clab("\status{0N}")
   tourcw(pi/4,"$B$");
   tourcw(-3*pi/4,"$A$",dotted)
 }
}
move right 1.5*hu
{clab("\status{00}")
 tourccw(pi/4,"$A$",dotted)
 {ln(0,-1,"$B$",0)
   clab("\status{0N}")
   tourcw(pi/4,"$B$");
   tourcw(-3*pi/4,"$A$",dashed)
   {alncw(pi/2,-pi/2,0,1,"$A$",dashed)}   
 }
}
.PE
\end{toimage}
\showgraph
In the case of the left automaton, only {\tt 1} will ever get printed.
In the case of the right
automaton, {\tt 2} will be printed when some messages arrives on~$B$
and then on~$A$.
Both automata lead to correct implementations of the semantics. However
the second automata looks a better choice than the first one, since it
yields more program behaviors.


\section{Runtime definitions}

\label{implementation}
\subsection{Basics}

Names are the basic values of the join-calculus, and thus any
implementation of the join-calculus must supply a runtime
representation for them. For instance, a name can be sent on some
appropriate channel. At runtime, we must indeed send something.

However, names that are defined together in the same join definition
interact when matching is tested for and performed.  Moreover, by the
very idea behind the join-calculus, matching is the only
synchronization primitive.  In other words, only names that are
defined by the same join definition have some kind of interaction that
is of the runtime system responsibility.


This makes possible and desirable to compile a source definition into
a runtime ``definition'', a single vector structure that groups all
the names defined in a given definition.  Names must still exist as
individuals, they can be represented as infix pointers into their
definition (as in {\tt join}), or as a definition pointer and an index
(as in {\tt jocaml}).

Both the {\tt join} and {\tt jocaml} systems implement the automata of
the previous section.  However, they do so in quite different manners.
The former focuses on minimizing runtime testing, while the latter
involves a systematic runtime testing of the current status at every
message arrival.

\subsection{Definitions in {\tt join}}

Runtime definitions are vector structures.  Every name defined in a
definition occupies two slots in the vector structure.  The first
entry holds a code pointer that stands for the name itself, while the
second entry holds a pointer to a queue of pending messages, queues
being organized as linked lists.  Runtime definitions include
additional slots that hold the values of the variables that are free
in guarded processes. This technique resembles much the one used by
the SML/NJ compiler~\cite{Appel-compiling} to represent mutually
recursive functions.  Message sending on name~$x$ is performed by
stacking message values and then calling the code for name~$x$.  This
code is retrieved by dereferencing twice the infix pointer that
represents~$x$ at runtime.


However, there is a big difference between mutually recursive
functions and join definitions.  The code for name~$x$ is automaton
code that reacts to the arrival of a new message on that name.  The
compiler issues various versions of name code, one per possible status
of the definition that defines~$x$.  Typically, name code either saves
a message into the queue for~$x$ (in the non-matching case), or
retrieves messages from other queues before firing a guarded process
(in the matching case). In all cases, definition status may then
need an update, which is performed by updating all code entries in
the definition.


\subsection{Definitions in {\tt jocaml}}

In the {\tt jocaml} system, a name is a pointer to a definition plus
an index.  Definitions are still vectors structures, but there is only
one entry per name for message queues. Additionally, definitions hold
guarded closures (i.e. guarded process code plus free variable
values), a status field and a matching data structure.

Status field holds the current status of the definition as a
bit-field.  Each name status is encoded by a bit, using bit 1 for
\status{N} and bit 0 for \status{0}, and bit position is given by name
index.
%  The matching data structure is an array of maps from statuses to
%  clause indexes. 

Message sending is performed by calling a generic C~function from the
``join'' library, taking message value, a definition pointer and a
name index as arguments.  When a message is received on name~$x$, the
bit for~$x$ is checked in the current status bit-field. If the bit is
set, some messages on name~$x$ are already present. Thus, definition
status does not change. Since the current status before message
sending is guaranteed to be a non-matching one, the message is queued
and the function is exited.

Otherwise, the current definition status is searched in the matching
structure for~$x$. This matching structure is an array of pattern
encoding, guarded process index pairs. Pattern encodings are
bit-fields, just like status encodings.
 corresponding to a join pattern containing name~$x$, from which name~$x$ has been removed. Using a sequential
search by a bitwise "and" with each pattern encoding, the current
status can be identified as matching or non-matching in at most $N_x$
tests, where $N_x$ is the number of clauses whose pattern contains~$x$.

If no match is found, the automaton state is updated and the message
value is queued in the queue for $x$.  Otherwise, a guarded process
index has been found, and is used to retrieve the associated guarded
closure.  Arguments to the guarded process are extracted from the
queues identified by the matching status found.  Status is updated at
the same moment (when a queue becomes empty a bit is erased).  Finally
the guarded process is fired.

Therefore, the performance of this technique much relies on fast
comparisons and modifications of definition statuses.  The best result
is achieved when statuses are encoded by machine integers.  In that
case, the number of names that a definition can define is limited by
the integer size of the hoisting Objective Caml system (which
typically is 31 or 63~bits).  If this is not considered enough, then
statuses have to be encoded using several integers or one string. Both
kinds of status encodings can be mixed, using integers for small
definitions and strings for larger ones. However, in the current {\tt
  jocaml} system, we use a single integer to hold status, and a
technique (described in section~\ref{fusion}) is used to associate
several channels to a same bit in the status bit-field.

\section{The pragmatics of compilation}\label{opt}
\label{optimization}

This section is dedicated to optimizations that are first pertinent
for the {\tt join} technique and that  are performed by the
current version of the {\tt join} compiler.

We first introduce optimizations that improve the runtime behavior of
programs, both in speed and dynamic memory usage. Then, we show how to
reduce emitted code size.
We focus on optimizing definitions written in
object-oriented style, as described in the tutorial~\cite{Join}. As
this programming style proved quite frequent,
it is normal for us compiler writers to concentrate
our efforts on such definitions.

In this style, a definition is an objet. Object state is encoded by
asynchronous {\em state names}, while synchronous {\em methods} access
or modify object state.
For instance, given one state name $S$ and $n$ methods
$m_1$, $m_2$,\ldots $m_n$ taken in that order, we get:
\begin{verbatim}
let create(x_0) =
  let S(x) | m_1() = P_1(x)
  and S(x) | m_2() = P_2(x)
    ....
  and S(x) | m_n() = P_n(x) in
  S(x_0) | reply m_1,m_2,...,m_n to create
;;
\end{verbatim}
\noindent The synchronous call \verb+create(v)+ creates a new
object (i.e. a new $S$, $m_1$, $m_2$,\ldots $m_n$ definition)
and then sends back a $n$-tuple of its methods. Moreover, this call
initializes object state with the value~$v$.


\subsection{Refined status}
As a working example of an object-style definition, consider the
following adder:
\begin{verbatim}
let create(x_0) =
  let S(x) | get() = S(x) | reply x to get
  and S(x) | add(y) = S(x+y) | reply to add in
  S(x_0) | reply get,add to create
;;
\end{verbatim}

The adder has one state name $S$ and two methods $get$ and $add$.  We
then try to figure out some ``normal'' runtime behavior for it.  As
the initial \verb+S(x_0)+ is forked as soon as the adder definition
has been created, a highly likely initial situation is that there is
one message pending on $S$ and none on the other names.  Later on,
as some external agent invokes the $get$ or $add$ method, the message
pending on $S$ is consumed and the appropriate guarded process is
fired.  Either process quickly sends a message on $S$.  Thus, a likely
behavior is for the queue of $S$ to alternate between being empty and
holding one element, the queue being empty for short periods.  By some
aspects of the compilation of ``\verb+|+'' and of our scheduling
policy that we will not examine here, this behavior is almost certain.

As a matter of fact, this ``normal'' life cycle involves a blatant
waste of memory, queue elements (cons cells) are allocated and
deallocated in the general dynamic fashion, while the runtime usage of
these cells would allow a more efficient policy.  It is more clever
not to allocate a cell for the only message pending on $S$ and to use
the queue entry attributed to $S$~in the runtime definition as a
placeholder.  On the status side, this new situation is rendered by a
new ``\status{1}'' status.  Hence, $S$ now possesses a three valued
status: \status{0} (no message), \status{1} (one message in the queue
slot) or \status{N} (some messages organized in a linked list).  Thus,
assuming for the time being, that there may be an arbitrary number of
messages pending on $S$, the adder compiles into the automaton of
figure~\ref{adder} (adder names are taken in the order $S$, $get$,
$add$). This new automaton can be seen as an evolution of the $A$,
$B$, $C$ automaton of figure~\ref{abc}, with a slight change in
channel names.

\begin{figure}[th]
\begin{toimage}
.PS
[
circlerad=16*pt
hu=3 *cm
vu=3.5 *cm
clab("\status{000}")
{ln(-1,-1,"$get$",-1) ; clab("\status{0N0}")
  tourccw(-9*pi/10,"$get$")
  tourccw(pi/2,"$S$",dotted)
  {alncw(pi/4,-3*pi/4,1,1,"$S$",dotted)}
  {ln(1,-1,"$add$",1) ; clab("\status{0NN}")
    {alncw(3*pi/4,-pi/4,-1,1,"$S$",dashed)}
    {aln(pi/4,-3*pi/4,1,1,"$S$",dotted)}
    htccw(pi/3,-pi/2-pi/6,"$\begin{array}{l} get\\ add\end{array}$")
    tourccw(-5*pi/12,"$S$",dotted)
    tourcw(-7*pi/12,"$S$",dashed)
}}
{
  ln(0,-.67,"$S$",0) ; clab("\status{100}")
  {alncw(pi/2,-pi/2-pi/12,0,0.67,"$get$",dotted)}
  {aln(pi/2,-pi/2+pi/12,0,0.67,"$add$",dashed)}
  ln(0,-.67,"$S$",0) ; clab("\status{N00}")
  tourcw(5*pi/12,"$add$",dashed)
  tourccw(-3*pi/4,"$S$")
  tourccw(-17*pi/12,"$get$",dotted)
  {alncw(pi/2,-pi/2-pi/6,0,1.33,"$get$",dotted)}
  {aln(pi/2,-pi/2+pi/6,0,1.33,"$add$",dashed)}
}
{ln(1,-1,"$add$",1) ; clab("\status{00N}") ;
  tourcw(-pi/10,"$add$")
  tourcw(pi/2,"$S$",dashed)
  {aln(3*pi/4,-pi/4,-1,1,"$S$",dashed)}
  {ln(-1,-1,"$get$",-1)}}
]
.PE
\end{toimage}
\showgraph
\caption{\label{adder} Full automaton for the adder}
\end{figure}

Using the status~\status{1} not only spares memory, it also avoids
some of the runtime tests that compute post-matching status.
Basically, when a matching consumes the sole message pending on a name
with status~\status{1}, then the automaton already knows that this
name queue is empty.  For instance, when the automaton of
figure~\ref{adder} is in the \status{100} status and that a message
arrive on either one of the two methods, then the appropriate process
is fired and status goes back to \status{000} without any runtime
test.  By contrast, when the automaton is in the \status{00N} status
and that a message arrive on $S$, the second guarded process is fired
immediately, but a test on $add$ queue is then performed: if this
queue is now empty then status goes back to \status{000}, otherwise
status remains \status{00N}.  Receiving a message on $S$ when status
is \status{0NN} is a bit more complicated. First, the automaton
chooses to consume a message pending on either one of the two methods
and to fire the appropriate process (figure~\ref{adder} does not
specify this choice).  Then, the queue of the selected method has to
be tested, in order to determine post-matching status.

Status \status{1} is easy to implement using the {\tt join}
compilation technique. The compiler issues different method codes for
\status{100} and \status{N00}, and different codes can find
$S$~argument at different places.  Implementing status \status{1} in
{\tt jocaml} would be more tricky, since the encoding of states using
bit-patterns would be far less straightforward than with
\status{0}/\status{N} statuses only.  As a consequence, the {\tt
  jocaml} compiler does not perform the space optimization described
in this section.

\subsection{Taking advantage of semantical analysis}

The automaton of figure~\ref{adder} has a \status{N00} status, to
reflect the case when there are two messages or more pending on $S$.
However, one quite easily sees that that status \status{N00} is
useless.  First, as $S$~does not escape from the scope of its
definition, message sending on~$S$ is performed at three places only:
once initially (by \verb+S(x_0)+) and once in each guarded process.
Thus, there is one message pending on $S$ initially.  A single message
pending on~$S$ is consumed by any match and the process fired on that
occasion is the only one to send one message on $S$. Therefore, there
cannot be two messages or more pending on $S$. As a consequence the
full automaton can be simplified by suppressing the \status{N00} node
and every edge that starts from it or leads to it.

In particular, there is no more $S$-labeled edge starting from node
\status{100}. In the {\tt join} implementation this means that 
the code entry for $S$ needs not be updated when going from status
\status{000} to \status{100}. This entry is simply left as it is.
Symmetrically, the code entry for $S$ will not have to be restored when
status goes back to \status{000} after matching.




Another important usage of semantical analysis is determining which names
are state names.  For a given definition, the output of the
analyzer is a status set~${\cal S}$, which is a safe approximation of
the actual runtime statuses of that definition.  State names are the
asynchronous names such that all statuses in ${\cal S}$ give them the
status \status{0}~or~\status{1}.

The current {\tt join} compiler includes a rudimentary
name usage analyzer, which suffices for object definitions given in
the style of 
the $S$, $m_1$, $m_2$, \ldots, $m_n$ definitions , where all state
variables are asynchronous and do not escape from the scope of their
definition.

An promising alternative would be to design an ad hoc syntax for
distributed objects, or, and this would be more ambitious, a full
object-oriented join-calculus. Then, the state variables of
object-definitions would be apparent directly from user programs.



\subsection{Avoiding status space explosion}\label{space}

Consider any definition that defines $n$ names.  Ignoring
\status{1}~statuses, the size of the status space of a given
definition already is $2^n$.  The size of the non-matching status
space is thus bounded by $2^n$.  As a full automaton for this
definition has one state per non-matching status, status space size
explosion would be a real nuisance in the case of the {\tt join}
compiler.  In particular, there are $n$~times the number of
non-matching statuses automaton code entries to create.

Unfortunately the exponential upper bound is reached by practical
programs, as demonstrated by the general object-oriented definition
given at the beginning of this section~\ref{opt}.  In that case, all
definition statuses such that $S$ has the \status{0} status are
non-matching.  In such a situation, using runtime testing, as {\tt
  jocaml} does, is not that much a penalty, when compared to code size
explosion.

We thus introduce dynamic behavior in the automata.  We do so on a
name per name basis: the status of state names will be encoded by
automata states as before, whereas method statuses will now be
explicitly checked at runtime.  Thus, we introduce ``\status{?}'', a
new status, which means that nothing is known about the number of
messages pending on a name.  Additionally, we state that all methods
will have the \status{?} status, as soon as there is one message or
more pending on any of the methods.

This technique can be seen as merging some states of
the full automaton compiled by considering complete status
information into new states with \status{?} statuses in them.

For instance, in the adder example, we get the automaton of
figure~\ref{adder2}, where the three statuses \status{0N0},
\status{0NN} and \status{00N} of figure~\ref{adder} merge into the new
status~\status{0??}.  (Note that we also take advantage of name usage
analysis to delete status \status{N00}.)

\begin{figure}[th]
\begin{toimage}
.PS
clab("\status{000}")

{{iln(-.4,-.4) ; "$\begin{array}{l}get\\add\\[.5em]\end{array}$"}
 ln(-1,-1,"",0) ; clab("\status{0??}")
 tourccw(pi,"$\begin{array}{l}get\\add\\[.5em]\end{array}$")
 tourcw(pi/4-pi/12,"$S$",dashed)
 tourccw(pi/4+pi/12,"$S$",dotted)
 {alncw(pi/4,-3*pi/4-pi/12,1,1,"$S$",dotted)}
 {aln(pi/4,-3*pi/4+pi/12,1,1,"$S$",dashed)}
}

{ln(1,-1,"$S$",0)
   clab("\status{100}")
   {aln(3*pi/4,-pi/4+pi/12,-1,1,"$add$",dashed)}
   {alncw(3*pi/4,-pi/4-pi/12,-1,1,"$get$",dotted)}
}
.PE
\end{toimage}
\showgraph
\caption{\label{adder2} Final automaton for the adder}
\end{figure}

Information on where runtime testing has to be performed can be
inferred from the diagram of figure~\ref{adder2}.  For instance,
assume that current status is \status{0??} and that a message arrives
on~$S$.  Since there is at least one message pending on a method, a
matching will occur.  Tests are needed though, before matching to
determine the matching clause, and after matching to determine
post-matching status.  Abstractly, the first series of tests changes
the \status{?}  statuses in either \status{0} or \status{N} statuses,
while the second series checks if there are still messages pending on
names with \status{?} status.  We are still investigating on how to
organize these tests efficiently without producing too much code
(see~\cite{Augustsson85,Maranget92} for a discussion of the size of
such code in the context of compiling ML~pattern-matching).

By contrast, when status is \status{100} and that a message arrives on
$get$ or $add$, then the corresponding matching is known immediately
and the message pending on $S$ is consumed. Then, the queue for $S$ is
known to be empty and status can be restored to \status{000} with no
runtime testing at all.  As message arrival order is likely to be
first one message on $S$ and then one message on $get$ or $add$ the
final automaton of figure~\ref{adder2} responds efficiently to more
frequent case, still being able to respond to less frequent cases (for
instance, two messages on methods may arrive in a row).  Furthermore,
when trouble is over, the automaton has status \status{000} and is
thus ready for the normal case. In this example, a penalty in code
size is paid for improving code speed in the frequent, ``normal''
case, whereas this penalty is avoided in non-frequent cases, which are
treated by less efficient code.

We introduced a \status{?} status on a name per name basis.  However,
there are other choices possible: {\em a priori}, there are many ways
to merge full automata states into final automata states. However, if
one really wants to avoid status space explosion the final automaton
should be constructed directly, without first constructing the full
automaton.  Adopting our per name \status{?} status makes this direct
construction possible. Additionally, the \status{?} status can be used
by the simple static analyzer as a status for the names it cannot
trace (e.g. names that escape their definition scope).  This
dramatically decreases the size of analyzer output and its running
time.

\section{Optimizing further}
\label{fusion}

We here describe a simple transformation on
join definitions, which does not rely on a full semantical analysis
(such as name usage analysis),
but only on a simple, local, syntactical analysis of join-patterns.


Let us take a simple example:
\begin{verbatim}
let S(x) | m_1(y) = P_1(x,y)
and S(x) | m_2(y) = P_2(x,y)
  ....
and S(x) | m_n(y) = P_n(x,y)
;;
\end{verbatim}

In this example, a match occurs only when there are messages pending
both on $S$ and on one of the methods $m_1$, $m_2$,\ldots Thus, from the
synchronization
point of view, all the methods are equivalent. And indeed, we can
regroup them into one, single method channel $m$ by
transforming this join definition into:

\begin{verbatim}
let S(x) | m(p,y) = P[p] (x,y);;
let m_1(y) = m(1,y);;
let m_2(y) = m(2,y);;
 ....
let m_n(y) = m(n,y);;
\end{verbatim}
Where $P$ is the vector of processes $[P_1,P_2,...,P_n]$.

Methods $m_1$ $m_2$,\ldots are now simple wrappers.  Method $m_i$ now
calls $m$ with an additional $i$~argument, which basically is the
index of $P_i$ in array $P$.  At this point, we must emphasize that we
describe this technique as a source to source transformation only for
clarity. However, the produced source code may not be correct with
respect to the join type system, when the types of methods are
different.  Anyhow, this optimization is implemented using ad hoc
mechanisms, this both improves efficiency and solves the typing
problem.

Currently, this optimization is performed by the {\tt jocaml}
compiler.  This leads to a new interpretation of name indexes by the
``join'' library.  The least significant bits in name indexes still
encode names that actually take part to synchronization (here $S$ and
$m$), while their most significant bits (which were previously unused)
now encode the extra $i$~argument.  This yields two benefits. First,
the number of status checks decreases, as the number of matching
statuses decreases.  Second, the number of channels that can be
defined in one definition can now exceed the hosting system integer
size, provided some names can be grouped together for synchronization.

In the {\tt join} compiler, this technique might be used to reduce
automata size, since it lowers the number of non-matching statuses, by
reducing the number of synchronizing names.  Code entries for methods
$m_1$, $m_2$,\ldots would still be contained in the definition
structure, they would only stack the index of the process to fire, and
then call the code for method $m$. Moreover, they do not need to be
updated after each transition of the automaton.


Finally, this technique can also be applied to more complex
synchronizations.  Given a definition that defines names $x_1$, $x_2$,
\ldots, $x_n$, using patterns $J_1$, $J_2$, \ldots $J_m$.  We say that
two names are equivalent, when swapping them in the patterns yields
the same set of patterns.  We then replace equivalent names by a
single name, plus an index.

Consider the following definition
\begin{verbatim}
let S_1(x) | m_1(y) = P_1(x,y)
and S_1(x) | m_2(y) = P_2(x,y)
and S_2(x) | m_1(y) = P_3(x,y)
and S_2(x) | m_2(y) = P_4(x,y)
;;
\end{verbatim}
Then the set of defined names $\{S_1, S_2, m_1, m_2\}$ can be
partitioned into $\{S_1, S_2\}$ and $\{m_1, m_2\}$.
Then, the above program can be transformed into:
\begin{verbatim}
let S(p,x) | m(q,y) = P[p,q] (x,y);;
let m_1(y) = m(1,y);;
let m_2(y) = m(2,y);;
let S_1(y) = S(1,y);;
let S_2(y) = S(2,y);;
\end{verbatim}
(with $P[1,1] = P_1$, $P[1,2] = P_2$, $P[2,1] = P_3$ and $P[2,2] = P_4$)


\section{Conclusion and future work}\cutname{conclusion.html}
\label{conclusion}

In join-calculus, a name definition, all receptors on that name and all
possible synchronizations on that name are grouped altogether in a single
join definition. This enables the compilation of synchronization
between concurrent or even distributed processes, using finite-state
automata.
In the distributed case, a message transport phase to the
machine that currently hosts the join definition (and hence the
automaton) is first performed.  This strengthens our point of view
that the join-calculus is the core of a distributed programming
language that can be compiled in practice, mainly because it restricts
reception on a channel to statically known parts of the program.  The
same argument applied to \`a la ML polymorphic typing
in~\cite{join-typing-97}.

\begin{cutflow}{Benchmarks}\cutname{benchmarks.html}
\aname{bench}{}
\begin{table}[ht]
\begin{center}
\begin{tabular}{l@{\quad}r@{\quad}r@{\quad}r@{\quad}r@{\quad}r}
& \texttt{fib} &
\texttt{afib} &
\texttt{pat} &
\texttt{qsort} &
\texttt{count}\\ \hline
\texttt{join} &  32.0 & 14.5 & 37.2 & 9.9 & 16.4 \\
\texttt{jocaml} & 5.7 &  3.5 & 5.4 & 1.4 & 4.2 \\
\texttt{Bologna} & 11.9 & 6.2 &  9.4 & 16.8 & 5.3 \\
\end{tabular}
\end{center}
\caption{\label{perf}Some performance measures (wall-clock time, in seconds)}
\end{table}
Benckmarks sources are available on the \footahref{http://join.inria.fr/speed/}{web}
\end{cutflow}
Taking a few benchmarks (see table~\ref{perf}, or \ahrefloc{bench}{here})
as a set of sensible join programs, both the
{\tt join} and the {\tt jocaml} pattern matching compilation schemes
prove adequate.  In particular, none of the two schemes falls into the
pitfall associated to the compilation technique used.

In the {\tt join} case, one can be afraid of code size, the technique
exposed in section~\ref{space} successfully avoids code size explosion
in practical cases.  The {\tt jocaml} technique appears expensive in
runtime checks and thus {\em a priori} produces slow code.  We choose
such a scheme of implementing automata using generic code, because it
can be implemented simply by adding code to the Objective Caml
bytecode interpreter.  Using bytecode specialized for automata
manipulation would have implied more important modifications of the
Objective Caml bytecode interpreter. Moreover, the {\tt jocaml} system
runs faster than the {\tt join} system, even for pure join programs,
showing that its weaker compilation of join definitions is more than
compensated by its total integration in the Objective Caml system.

Comparison with the Bologna implementation~\cite{Bologna} of the
join-calculus is also instructive.  This system also produces
bytecode, which is interpreted by a C~program.  It proves faster than
{\tt join} and slower that {\tt jocaml} on most of the examples.
Taking a glance at the Bologna source code reveals that it uses a
scheme very similar to the one of {\tt jocaml}, with two slight
differences.  First, status is systematically encoded as an array of
integers.  Second when a message arrives on a name $x$ with an empty
queue, all patterns are tested (whereas in {\tt jocaml} only the
patterns that contain $x$ are tested).

Performance of a given join system depends on many factors. In
particular, scheduling policy and message queue management have a
dramatic impact on it. Furthermore, a policy that gives good results
on one benchmark can be defeated by another.  For these reasons, we
cannot tell which compilation technique is the best by comparing
different implementations.



Clearly, we now need to integrate all our compilation techniques in
the same system, in order to compare them more thoroughly and to
experiment further.  However, the definition of reactivity status and
the automata of section~\ref{automatas} provide a sound
basis for these future investigations.  Apart from future language
development and fully implementing the failure semantics of the
join-calculus, we also plan to investigate more on the implementation
of threads, attempting to minimize thread suspension and creation.



\bibliography{bib}
\bibliographystyle{abbrv}


\end{document}

%\subsection{Playing with the runtime scheduler}
%
%Concentrating on the {\it object style} of programming, lots of
%methods do not modify the automaton status, but only read or modify
%state names (such as {\tt set} and {\tt get}), with only little
%non-blocking computations. Most of the time, these methods are never
%interrupted, and thus behave as in critical sections from the
%threads scheduler point of view.
%
%As a consequence, when such a method is called, an optimization would
%be not to modify the automaton status at all, since the method will
%end by giving back its original status to the automaton. To prevent
%the automaton to receive other messages from other methods while in an
%incoherent state, these methods are executed in a critical section.
%
%This optimization relies on a simple analysis of methods, which can
%also be used to decrease the number of forked threads, since such
%methods do not need to be executed in a forked thread. 
