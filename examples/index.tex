\documentclass{article}
\usepackage{html}

\begin{document}
Here, you find some examples of using \hevea.
\begin{itemize}
\item The \oneurl{Makefile}
\item A simple article,  \oneurl{a.tex} and \oneurl{a.html}. It is
generated with the french option set:
\begin{verbatim}
# htmlgen -francais a.tex
\end{verbatim}
\item The test suite: \oneurl{suite.tex} and \oneurl{suite.html}
\item A full article cut into pieces , with images (source is \oneurl{pat.tex}
and output is \oneurl{patbis.html}). To yied this result, \verb+gpic+ is first
applied to \verb+pat.tex+, yielding \verb+temp.tex+; \LaTeX\ and
\texttt{bibtex} are then used to produce the appropriate \verb+tmp.aux+ and
\verb+tmp.bbl+ files; finally \hevea\ and \hacha\ are run as
follows:
\begin{verbatim}
# htmlgen pat.sty -e pat.def -o pat.html tmp.tex
# imagen pat
# hacha -o patbis.html pat.html 
\end{verbatim}
Notice that the \LaTeX\ style file \oneurl{pat.def} is not loaded and
that a simplified \oneurl{pat.sty} is loaded.

\end{itemize}
\end{document}