\documentclass{article}
\title{Examples of \hevea{} usage}
\begin{document}\maketitle

\section*{Examples from the distribution}

\begin{itemize}
\item The \ahrefurl{Makefile}

\item A simple article,  source is \ahrefurl{a.tex}.
You can compare \LaTeX{} output (\ahrefurl{a.dvi}) and \hevea{} output,
(\ahrefurl{a.html} and \ahrefurl{a.txt}). The \html{} output is
generated with the french option set:
\begin{verbatim}
# hevea -francais a.tex
\end{verbatim}
While the text output generated with:
\begin{verbatim}
# hevea -text -francais a.tex
\end{verbatim}

\item The test suite: \ahrefurl{suite.tex} and \ahrefurl{suite.html} (see
also \ahrefurl{suite.dvi} and \ahrefurl{suite.txt}).

\item A full article, cut into pieces , with images (source is
\ahrefurl{pat.tex}, output is \ahrefurl{pat.dvi}, 
\ahrefurl{pat.html} and \ahrefurl{pat.txt}). To yield this result, \verb+gpic+ is first
applied to \verb+pat.tex+, yielding \verb+tmp.tex+; \LaTeX\ and
\texttt{bibtex} are then used to produce the appropriate \verb+tmp.aux+ and
\verb+tmp.bbl+ files; finally \hevea{}, \hacha{} and \texttt{imagen}
are run as follows:
\begin{verbatim}
# hevea pat.hva -e pat.def tmp.tex
# hacha -o pat.html tmp.html
# imagen tmp
\end{verbatim}
Notice that the \LaTeX{} style file \ahrefurl{pat.def} is not loaded and
that a simplified \ahrefurl{pat.hva} is loaded.

Using option ``\texttt{-fix}'', \texttt{imagen} does not need to be
run explicitely~:
\begin{verbatim}
# hevea -fix pat.hva -e pat.def tmp.tex
# hacha -o pat.html tmp.html
\end{verbatim}

This example also serves as a test of \hacha{} advanced features,
direct access to the article \ahref{conclusion.html}{conclusion}
and \ahref{benchmarks.html}{performance figures}.

\item How to use the \verb+latexonly+ and \verb+toimage+ environments
inside other environments.
Source files are \ahrefurl{env.hva} and
\ahrefurl{env.tex}), output file is \ahrefurl{env.html}.
\texttt{gpic}, \hevea{} and \texttt{imagen}
are run as follows:
\begin{verbatim}
# gpic -t < env.tex > tenv.tex
# hevea env.hva tenv.tex -o env.html
# imagen env
\end{verbatim}

\item The ``\emph{Smile!}'' example, from the manual.
Files are \ahrefurl{smile.hva}, \ahrefurl{smile.tex} and
\ahrefurl{smile.html}, command sequence is~:
\begin{verbatim}
# gpic -t < smile.tex > tsmile.tex
# hevea smile.hva tsmile.tex -o smile.html
# imagen smile
\end{verbatim}

\item A few amstex primitives, \ahrefurl{amstex.tex},
\ahrefurl{amstex.dvi}, \ahrefurl{amstex.html} and \ahrefurl{amstex.txt}.
\hevea{} is invoked as:
\begin{verbatim}
# hevea [-text] amstex.tex
\end{verbatim}

\item The \texttt{graphics} package, \ahrefurl{graphics.tex},
\ahrefurl{graphics.dvi} and \ahrefurl{graphics.html}.
\hevea{} is invoked as:
\begin{verbatim}
# hevea graphics.tex
# imagen graphics
\end{verbatim} 
\end{itemize}

\section*{Other documents}
\begin{itemize}
\item The on-line \ahref{http://w3.edu.polytechnique.fr/informatique/TC/polycopie-1.6/index.html}{``\emph{polycopi�}''} of basic computer science at
�cole polytechnique.
\item The on-line version of my
\ahref{http://w3.edu.polytechnique.fr/profs/informatique/Luc.Maranget/}{course
notes} at �cole polytechnique (look at the ``\emph{TD}'' entries).
\item The on-line
\ahref{http://caml.inria.fr/ocaml/htmlman/index.html}{Objective Caml manual}.
\item The on-line \ahref{http://join.inria.fr/manual/}{join-calculus
manual}. Including \ahref{ftp://ftp.inria.fr/INRIA/Projects/para/join/jc-1.04-refman.txt.gz}{text} and \ahref{ftp://ftp.inria.fr/INRIA/Projects/para/join/jc-1.04-refman.info.tar.gz}{info} versions!
\item The on-line \ahref{http://coq.inria.fr/doc/main.html}{Coq proof assistant
 manual} was produced using a former version of
\hevea.
\item The on-line
\ahref{http://pauillac.inria.fr/\home{diaz}/gnu-prolog/manual/index.html}
{GNU-prolog manual}.
\item The on-line
\ahref{http://www.lifl.fr/\home{boulet}/softs/mldoc/index.html}{MlDoc}
manual. MlDoc is a tool that performs automatic documentation extraction for Objective
Caml.
\item The on-line
\ahref{http://loco.inria.fr/\home{georget}/clp\_fds/OnlineManual/index.html}{\texttt{clp(FD,S)}
manual}.
\end{itemize}
\end{document}
