
\documentclass {article}

\usepackage {mathpartir}
\usepackage {listings}
\usepackage {array}
\usepackage {url}
\newif \ifhevea 
%HEVEA \heveatrue
\ifhevea 
\usepackage {hevea}
\fi

\lstset {basicstyle=\tt}
\let \lst \verb

\title {Didier's Mathpartir}

%HEVEA \loadcssfile{x.css}
%HEVEA \loadcssfile{y.css}
%HEVEA \setstyles{background:green;color:black;padding:5}
%HEVEA \setHTMLstyle{H1.part}{background:green;color:white;font:helvetica}
%HEVEA \setLATEXstyle{section}{background:blue;color:white}
%HEVEA \setLATEXstyle{subsection}{background:blue;color:white}
%HEVEA \setHTMLstyle{H5}{background:cyan;color:black}
%HEVEA \setLATEXstyle{subparagraph}{background:magenta;color:white}

%\setHTMLstyle{H3}{background:blue;color:white;font:sans-serif}
%\setstyles{background:blue;color:white;font:sans-serif}
%\setstyles{backgound:green}


\begin{document}

\maketitle

\part{Only One Part}

\section {The inferrule macro}

The inferrule macro is designed to typeset inference rules.  It should
only\footnote {Even though the basic version may work in text mode,
we discourage its use in text mode; the star-version cannot be used in
text-mode} be used in math mode (or display math mode). 

The basic use of the rule is 
\begin{verbatim}
\inferrule
  {one \\ two \\ three \\ or \\ more \\ premisses}
  {and \\ any \\ number \\ of \\ conclusions \\ as \\ well}
\end{verbatim}
This is the rendering on a large page
\def \one {\inferrule
  {one \\ two \\ three \\ or \\ more \\ premisses}
  {and \\ any \\ number \\ of \\ conclusions \\ as \\ well}
}
$$
\ifhevea \one
\else
\fbox {\vbox {\advance \hsize by -2\fboxsep \advance \hsize by -2\fboxrule
       \linewidth\hsize
        $$\one$$}}
\fi
$$

However, the same formula on a narrower page will automatically be typsetted
like that:
\def \two {\inferrule
  {one \\\\ two \\ three \\ or \\\\ more \\ premisses}
  {and \\ any \\ number \\\\ of \\ conclusions \\\\ as \\ well}
}
$$
\ifhevea \two
\else
\fbox {\hsize 0.33 \hsize \vbox {$$\two$$}}
\fi
$$
An inference rule is mainly composed of a premisse and a conclusion. 
The premisse and the conclusions are both list of formulas where the
elements are separated by \verb"\\". 

Note the dissymetry between typesetting of the premisses and of
conclusions where lines closer to the center are fit first. 

A newline can be forced by adding an empty line \verb"\\\\"

\begin{tabular}{m{0.44\hsize}m{0.1\hsize}m{0.44\hsize}}
\begin{lstlisting}{Ocaml}
\inferrule 
   {aa \\\\ bb}
   {dd \\ ee \\ ff}
\end{lstlisting}
&
~~~
&
$\inferrule {aa \\\\bb}{dd \\ ee \\ ff}$
\\
\end{tabular}

\subsection {Single rules}

Single rules are the default mode. 
Rules are aligned on their fraction bar, as illustrated below:
$$
\inferrule {aa \\ bb}{ee} \hspace {4em} \inferrule {aa \\\\ bb \\ ee}{ee}
$$
If the premise or the conclusion is empty, then the fraction bar is not
typeset and the premise or the conclusion is centered:
$$
\begin{tabular}{m{0.45\hsize}m{0.1\hsize}m{0.45\hsize}}
\begin{lstlisting}{Ocaml}
\inferrule {}{aa} + 
\inferrule {aa \\\\ aa}{}
\end{lstlisting}
&
~~~~~~~~
&
$
\inferrule {}{aa} + 
\inferrule {aa \\\\ aa}{}
$
\\
\end{tabular}
$$
\paragraph{OOOOOH}
Use use \verb"{ }" instead of \verb"{}" to get an axiom for instance:
$$
\begin{tabular}{m{0.45\hsize}m{0.1\hsize}m{0.45\hsize}}
\begin{lstlisting}{Ocaml}
\inferrule { }{aa} + 
\inferrule {aa}{ }
\end{lstlisting}
&
~~~~~~
&
\mbox {$
\inferrule { }{aa} + 
\inferrule {aa}{ }
$}
\\
\end{tabular}
$$
\subparagraph{OOOOHOOOH}
The macro \lst"\inferrule" acceps a label as optional argument, which will
be typeset on the top left corner of the rule: 
\par
\begin{tabular}{m{0.45\hsize}m{0.1\hsize}m{0.45\hsize}}
\begin{lstlisting}{Ocaml}
\inferrule [yop]
   {aa \\ bb}
   {cc}
\end{lstlisting}
&
$\inferrule [Yop]{aa \\ bb}{cc}$
\\
\end{tabular}
\par\noindent
See section~\ref{options} for changing typesetting of labels. 
A label can also be placed next to the rule directly, since the rule is
centered: 
\par
\begin{tabular}{m{0.45\hsize}m{0.1\hsize}m{0.45\hsize}}
\begin{lstlisting}{Ocaml}
\inferrule 
   {aa \\ bb}
   {cc}
\quad (\textsc {Yop})
\end{lstlisting}
&
~~~~~~
&
$\inferrule{aa \\ bb}{cc} \quad (\textsc {Yop})$
\\
\end{tabular}

\subsection {Customizing}

By default, lines are centerred in inference rules. 
However, this can be changed by either \lst"\mprset{flushleft}"
or \lst"\mprset{center}". For instance, 

\begin{tabular}{m{0.44\hsize}m{0.12\hsize}m{0.44\hsize}}
\begin{lstlisting}{Ocaml}
$$\mprset{flushleft}
  \inferrule 
    {a \\ bbb \\\\ ccc \\ dddd}
    {e \\ ff \\ gg}$$ 
\end{lstlisting}
&
~~~~~
&
$$\mprset{flushleft} 
\inferrule {a \\ bbb  \\\\ ccc \\ dddd}{e \\ ff \\ gg}$$
\\
\end{tabular}

\noindent
Note that lines are aligned independently in the premise and the
conclusion, which are both themselves centered. In particular, 
left alignment will not affect a single-line premisse or conclusion. 


\subsection {Derivation trees}

To help writing cascades of rules forming a derivation tree, inference rules
can also be aligned on their bottom line. For this, we use the star-version:
$$
\begin{tabular}{m{0.45\hsize}m{0.1\hsize}m{0.45\hsize}}
\begin{lstlisting}{Ocaml}
\inferrule*
   {\inferrule* {aa \\ bb}{cc}
    \\ dd}
   {ee}
\end{lstlisting}
&
~~~~~~
&
$
\inferrule*
   {\inferrule* {aa \\ bb}{cc}
    \\ dd}
   {ee}
$
\\
\end{tabular}
$$
The star version can also take an optional argument, 
but with a different semantics. The optional argument is parsed by the 
\verb"keyval" package, so as to offer a set of record-like options: 
$$
\def \val {{\em v}}
\def \arraystretch {1.4}
\begin{tabular}{|>{\tt}c|p{0.7\hsize}|}
\hline
\bf key & \bf Effect for value {\val}
\\\hline\hline
before & Execute {\val} before typesetting the rule.
         Useful for instance to change the maximal width of the rule.
\\\hline
width &  Set the width of the rule to {\val}
\\\hline
narrower & Set the width of the rule to {\val} times \verb"\hsize".
\\\hline
left & Put a label {\val} on the left of the rule
\\\hline
Left & Idem, but as if the label had zero width. 
\\\hline
Right & As \verb"Left", but on  the right of the rule.
\\\hline
right & As \verb"left",  but on the right of the rule.
\\\hline
leftskip & Cheat by (skip negative space) {\val} on the left side.
\\\hline
rightskip & Cheat by {\val} on the right side of the rule.
\\\hline
vdots & Raise the rule by {\val} and insert vertical dots. 
\\\hline
\end{tabular}
$$
Here is an example of a complex derivation:
$$
\inferrule* [left=Total]
  {\inferrule* [Left=Foo]
     {\inferrule* [Right=Bar,
                       leftskip=2em,rightskip=2em,vdots=1.5em]
         {a \\ a \\\\ bb \\ cc \\ dd}
         {ee}
       \\ ff \\ gg}
     {hh}
  \\
  \inferrule* [lab=XX]{uu \\ vv}{ww}}
  {(1)}
$$
and its code
\begin{lstlisting}{Ocaml}
\inferrule* [left=Total]
  {\inferrule* [Left=Foo]
     {\inferrule* [Right=Bar,
                       leftskip=2em,rightskip=2em,vdots=1.5em]
         {a \\ a \\\\ bb \\ cc \\ dd}
         {ee}
       \\ ff \\ gg}
     {hh}
  \\
  \inferrule* [lab=XX]{uu \\ vv}{ww}}
  {(1)}
\end{lstlisting}

\def \L#1{\lower 0.4ex \hbox {#1}}
\def \R#1{\raise 0.4ex \hbox {#1}}
\def \hevea {H\L{E}\R{V}\L{E}A}
\def \hevea {$\mbox {H}\!_{\mbox {E}}\!\mbox {V}\!_{\mbox {E}}\!\mbox {A}$}


\subsection {Implementation}

The main macro in the implementation of inference rules is the one that
either premises and conclusions.  The macros uses two box-registers one
\verb"hbox" for typesetting each line and one \verb"vbox" for collecting
lines. The premise appears as a list with
\verb"\\" as separator. Each element is considered in turn typeset in a
\verb"hbox" in display math mode. Its width is compare to the space left on
the current line. If the box would not fit, the current horizontal line is
transferred to the vertical box and emptied. Then, the current formula can
safely be added to the horizontal line (if it does not fit, nothing can be
done). When moved to the vertical list, lines are aligned on their center
(as if their left-part was a left overlapped). At the end the vbox is
readjusted on the right. 

This description works for conclusions. For premises, the elements must be
processes in reverse order and the vertical list is simply built upside
down. 


\section {Other Options}

\label {options}

The package also defines \verb"\infer" as a shortcut for \verb"\inferrule"
but only if it is not previously defined. 

The package uses \verb"\TirName" and \verb"\RefTirName" to typeset labels,
which can safely be redefined by the user. The former is used for defining
occurrences ({\em ie.} in rule \lst"\inferrule") while the latter is used
for referencing ({\em ie.} in the star-version). 

The vertical space in \verb"mathpar" is adjusted by
\verb"\MathparLineskip". To restore the normal paragraph parameters in mathpar
mode (for instance for some inner paragraph), use the command
\verb"\MathparNormalpar".  
The environment uses \verb"\MathparBindings" to
rebind \verb"\\", \verb"and", and \verb"\par". You can redefine thus command
to change the default bindings or add your own. 


\end{document}
